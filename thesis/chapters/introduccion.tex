\chapter{Introducción}

La observación y medición del campo gravitatorio terrestre se remonta a los
orígenes de la Ciencia moderna. Galileo Galilei realiza las primeras
observaciones cuantitativas sobre la caída de los objetos en el Siglo~XIV. En
1687 Isaac Newton propone la existencia de la fuerza gravitatoria como
interacción entre toda la materia, describiendo una Ley Universal que la
gobierna en su famoso tratado \emph{Philosophiae Naturalis Principia
Mathematica}.

A lo largo de los siglos subsiguientes muchos científicos y muchas científicas
han realizado observaciones e interpretaciones acerca de las variaciones del
campo gravitatorio terrestre en distintas localidades. Jean Richer detecta en
1672 que los péndulos oscilaban a frecuencias ligeramente distintas en París
y en Guinea Francesa, dando nacimiento a la \emph{gravimetría}. Newton atribuye
esta variación a la forma ovalada de la Tierra. Es Pierre Bouguer quien
encabeza en 1735 una expedición para realizar cuidadosas mediciones de estas
variaciones a diversas latitudes.

Los experimentos llevados a cabo por Henry Cavendish alrededor de 1797
permitieron no solo estimar la densidad media de la Tierra, sino también un
valor para la constante de gravitación universal $G$, medición que se encuentra
muy de acuerdo al valor aceptado hoy en día.

Las primeras aplicaciones de mediciones gravimétricas a la geología
vienen de la mano de John Pratt y George Airy en la década de 1850, proponiendo
dos hipótesis opuestas acerca de cómo las partes rígidas de la corteza y el
manto \emph{flotan} sobre un substrato fluido, dando origen al concepto de
\emph{isostasia}.

Para finales del Siglo~XIX y durante el Siglo~XX, los avances tecnológicos
permitieron perfeccionar los instrumentos a través de los cuales se pueden
realizar observaciones del campo gravitatorio terrestre. Vale destacar los
desarrollos de Vening Meinesz, Roland von Eötvös, LaCoste y Romberg
\citep{blakely1995}.
Estos abrieron las puertas a la aplicación de las observaciones gravimétricas
a múltiples problemas geológicos y geofísicos. Integrando estas mediciones con
diversas metodologías de las Geoiencias, permiten un amplio conocimiento sobre
las estructuras y cuerpos debajo de la superficie terrestre.

El arribo del Siglo~XXI nos recibe con misiones espaciales
diseñadas para situar satélites especialmente dedicados a la medición del campo
gravitatorio terrestre de manera global y a lo largo del tiempo.
Este tipo de observaciones pueden integrarse con las obtenidas en la superficie
terrestre, sobre la superficie marítima o desde el aire para dar mayor
completitud a los resultados e interpretaciones.

Las aplicaciones de los conocimientos que la gravimetría produce se extienden
desde la exploración de recursos naturales, mayor comprensión sobre el origen
de los fenómenos de nuestro planeta hasta la observación y mitigación de los
cambios ambientales, climatológicos y biológicos.

En esta Tesis presentamos los avances realizados por el autor a lo largo de su
Doctorado, vinculados al modelado de los campos gravitatorios y al
procesamiento de las observaciones gravimétricas.
A lo largo del texto desarrollaremos una descripción detallada de dos nuevas
metodologías.

La primera consiste en un algoritmo para el cómputo de campos gravitatorios
generados por geometrías definidas en coordenadas esféricas conocidas como
\emph{teseroides} (o prismas esféricos) cuya densidad varía a lo largo de la
dirección radial.
El principal interés sobre este nuevo método reside en su potencial aplicación
al modelado de estructuras regionales o continentales que presenten variaciones
de densidad en su profundidad, como pueden ser grandes cuencas sedimentarias
o determinadas regiones del manto litosférico o astenosférico.
Dicha metodología y sus resultados han sido publicados en \emph{Geophysical
Journal International} bajo el título ``Gravitational field calculation in
spherical coordinates using variable densities in depth'' \citep{soler2019}.
Una preimpresión del artículo se encuentra disponible bajo licencia Creative
Commons Atribución 4.0 Internacional en
\href{https://eartharxiv.org/}{EarthArXiv}:
\url{https://doi.org/10.31223/osf.io/3548g}.

La segunda metodología que procederemos a describir consiste en un algoritmo
para la interpolación de datos gravimétricos basado en el concepto de fuentes
equivalentes que incorpora elementos de aprendizaje automático (\emph{machine
learning}).
Más precisamente, presentamos las fuentes equivalentes potenciadas por
gradiente (\emph{gradient-boosted equivalent sources}), las cuales permiten
realizar interpolaciones de grandes cantidades datos de cualquier campo
potencial armónico, reduciendo considerablemente los requerimientos
computacionales.
Esta nueva metodología
ha sido publicada en \emph{Geophysical Journal International} bajo
el título ``Gradient-boosted equivalent sources'' \citep{soler2021}, junto con
todas las pruebas realizadas sobre datos sintéticos y una aplicación sobre un
gran conjunto de datos gravimétricos sobre Australia.
Una preimpresión del artículo se encuentra disponible bajo licencia Creative
Commons Atribución 4.0 Internacional en
\href{https://eartharxiv.org/}{EarthArXiv}:
\url{https://doi.org/10.31223/X58G7C}.

En tercer lugar, haremos una descripción de las herramientas disponibles en las
librerías de Fatiando a Terra \citep{uieda2013}: un proyecto de código abierto
para Geofísica, en el cual el autor de esta Tesis ha realizado múltiples
contribuciones.
Recorreremos los orígenes e historia del proyecto, describiremos las
herramientas desarrolladas en él y algunos de los métodos implementados en
ellas, mostraremos algunos ejemplos sencillos y pondremos en evidencia el
impacto de Fatiando a Terra en la comunidad geocientífica internacional y en
cómo este tipo de software resulta indispensable para la reproducibilidad de
las investigaciones científicas.
Además, expondremos las prácticas utilizadas para el desarrollo de estas
herramientas, tales como el manejo de controladores de versiones, el uso de
pruebas de software, la utilización de integración continua y la redacción de
documentación, entre otras.

Por último finalizaremos con un Capítulo donde se detallan las
conclusiones de todo lo desarrollado a lo largo de esta Tesis de manera global,
recorriendo cada una de las metodologías aquí introducidas y cómo se
vincula el proyecto Fatiando a Terra con las mismas.
