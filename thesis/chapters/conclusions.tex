\chapter{Conclusiones}

% que se hizo

% resultados obtenidos

% integracion ciencia con desarrollo de software

% aprendizaje de mejores practicas

% ciencia reproducible

% planes a futuro

% implementacion de tesseroides de densidad variable y fuentes equivalente
% gradient boosted en Harmonica

Durante el desenvolvimiento de esta Tesis Doctoral se desarrollaron dos nuevas
metodologías para el modelado y procesamiento de datos gravimétricos que pueden
ser aplicadas a un amplio espectro de problemáticas geofísicas que pueden
abarcar desde investigaciones científicas hasta exploración de recursos
naturales.
Cada una de estas metodologías han sino implementadas mediante software
a través del lenguaje de programación Python y se encuentran disponibles bajo
licencias de código abierto.

\vspace{1em}

La primer metodología consiste en un modelo directo que permite calcular los
campos gravitacionales generados por tesseroides cuya densidad puede expresarse
como una función continua dependiente de la coordenada radial.
Este nuevo método resuelve numéricamente las integrales involucradas mediante
una \acl{GLQ} de segundo orden aplicada en conjunto con dos algoritmos de
discretización:

\begin{itemize}
    \item un algoritmo de discretización adaptativa bidimensional que divide al
        tesseroide en las direcciones longitudinales y latitudinales en función
        de la distancia entre el centro del tesseroide y el punto de cómputo, y
    \item  un algoritmo de discretización basado en la densidad que lo
        fracciona en la dirección radial en los lugares donde toma lugar la
        \emph{máxima variación de densidad}.
\end{itemize}

Estos algoritmos cuentan con dos parámetros que controlan de manera indirecta
la cantidad de subdivisiones que se llevan a cabo: el ratio distancia-tamaño
$D$ y el ratio delta $\delta$, respectivamente.
Los valores que estos asumen poseen un impacto sobre el tiempo de cómputo
necesario para finalizar los cálculos y sobre la precisión de los resultados.

Con el objetivo de hallar valores óptimos de $D$ y $\delta$ que minimicen el
tiempo de cómputo mientras alcanzan precisiones aceptables, llevamos a cabo
comparaciones empíricas entre los resultados numéricos y las soluciones
analíticas para cascarones esféricos con densidades lineales, exponenciales
y sinusoidales.
Como resultado, hallamos que al utilizar valores de $D$ de 1 y 2.5 para el
potencial gravitatorio y para su gradiente, respectivamente, la integración
numérica alcanza precisiones por debajo del 0.1\% para cascarones de diferentes
espesores y sobre puntos de observación situados en diversas localizaciones.
De manera análoga, la elección de un ratio $\delta$ de 0.1 alcanza la misma
precisión para diferentes funciones de densidad, a excepción de funciones
sinusoidales con altas frecuencias en cuyo caso alcanza una precisión de 1\%
con respecto a las soluciones analíticas.
Además, podemos concluir que las elecciones de $D$ y $\delta$ no poseen una
relación significativa entre sí y por ende la elección de sus valores puede
realizarse de forma independiente.

Si bien la incorporación del algoritmo de discretización basado en la densidad
introduce un mayor costo computacional, esta metodología presenta un alto grado
de flexibilidad: es posible utilizar cualquier tipo de función continua sin
necesidad de modificar la implementación del algoritmo.
Esta característica hace que esta metodología pueda ser ampliamente utilizada
para diferentes fines, sin necesidad de realizar de limitarse a supuestos
previos.

Finalmente, esta nueva metodología fue aplicada para el modelado directo de la
Cuenca Neuquina ubicada al Norte de la Patagonia Argentina.
Se ha calculado el efecto gravitatorio del paquete sedimentario de la cuenca
tomando en consideración la compactación del relleno, demostrando que existen
una diferencia apreciable en comparación a asumir valores constantes de
densidad.

\vspace{1em}

La segunda metodología consiste en un algoritmo que permite realizar
interpolaciones de grandes cantidades de datos gravimétricos o magnéticos
mediante la utilización de \emph{fuentes equivalentes potenciadas por
gradiente}.
Este nuevo método construye conjuntos más pequeños de fuentes equivalentes
mediante la utilización de ventanas solapadas, cuyos coeficientes son
determinados de forma iterativa mediante un algoritmo de mínimos cuadrados
amortiguados.
De esta forma, las \emph{fuentes equivalentes potenciadas por gradiente} son
capaces de transformar un gran problema con altos requisitos de memoria
computacional en pequeños problemas más sencillos de resolver que son
acumulados de forma iterativa.

Mediante sucesivas pruebas sobre datos sintéticos, hemos sido capaces de
demostrar que esta nueva metodología es capaz de alcanzar precisiones similares
a las obtenidas mediante fuentes equivalentes regulares, reduciendo el tiempo
de cómputo de forma considerable y permitiendo interpolar grandes cantidades de
datos.
También hallamos que un solapamiento del 50\% entre ventanas contiguas produce
un balance óptimo entre tiempo de cómputo y precisión de las predicciones.
Además, hemos demostrado que aleatorizar el orden en el cual se lleva a cabo la
iteración de las ventanas es fundamental para obtener resultados más precisos
y acelerar la convergencia del algoritmo.
Con el objeto de maximizar la precisión del método, recomendamos elegir el
máximo tamaño de ventana que genere matrices Jacobianas que puedan almacenarse
en memoria.

En adición a esta metodología, presentamos una nueva estrategia para ubicar las
fuentes equivalentes, que denominamos \emph{fuentes promediadas por bloques}.
Esta consiste en dividir la región de estudio en bloques de igual tamaño
y ubicar una fuente en la posición horizontal media de las observaciones que
caen dentro de cada bloque.
De esta forma, la cantidad de fuentes equivalentes es mucho menor si comparamos
con otras estrategias, como ubicar una fuente debajo de cada observación o en
una grilla regular.

Mediante la utilización de datos sintéticos, hemos comparado las predicciones
que se obtienen al utilizar cada distribución de fuentes equivalentes,
incluyendo también diversas formas de determinar sus profundidades.
Nuestros resultados indican que las \emph{fuentes promediadas por bloques}
reducen el costo computacional del método, alcanzando el mismo nivel de
precisión que al utilizar distribuciones de fuentes clásicas.
A partir de nuestra experiencia, recomendamos elegir un tamaño de bloque del
mismo orden que la resolución de la grilla sobre la cual se interpolarán los
datos.

Hemos observado que las diferentes estrategias para asignar la profundidad de
las fuentes no modifican significativamente la precisión de las
interpolaciones.
A excepción de las \emph{profundidades variables}, que producen resultados
ligeramente más precisos.
Sin embargo, dado que involucran mayor cantidad de hiperparámetros, su uso
puede conllevar un aumento en el costo computacional durante su determinación
mediantes métodos como la validación cruzada.
Por esta razón, recomendamos la utilización de \emph{profundidades constantes}
o \emph{profundidades relativas}.

Dado que las \emph{fuentes equivalentes potenciadas por gradiente} no realizan
ninguna suposición sobre las ubicaciones de las fuentes, es posible utilizar
este método junto con \emph{fuentes promediadas por bloques}, reduciendo los
requerimientos de memoria computacional por dos medios independientes.

Finalmente, hemos aplicado las \emph{fuentes equivalentes potenciadas por
gradiente} junto con \emph{fuentes promediadas por bloques} para interpolar una
colección de datos gravimétricos sobre Australia con más de 1.7 millones de
puntos sobre una grilla regular a altitud constante con una resolución de
1 minuto de arco, aproximadamente.
El proceso de interpolación fue llevado a cabo en una computadora personal
modesta, utilizando menos de 16~Gigabytes de memoria computacional.
