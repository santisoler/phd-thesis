\chapter{Tesseroides de densidad variable}
\label{cha:tesseroids-variable-density}

Este capítulo es una traducción al Español del artículo titulado
\emph{Gravitational field calculation in spherical coordinates using variable
densities in depth} escrito por Santiago R. Soler, Agustina Pesce, Mario E.
Gimenez y Leonardo Uieda, y publicado en \emph{Geophysical Journal
International} en Junio de 2019 \citep{soler2019}.
Una preimpresión del artículo se encuentra disponible bajo licencia Creative
Commons Atribución 4.0 Internacional en
\href{https://eartharxiv.org/}{EarthArXiv} (doi:
\href{https://doi.org/10.31223/osf.io/3548g}{10.31223/osf.io/3548g}).

\section{Resumen}

Presentamos una nueva metodología para calcular los campos gravitatorios
generados por tesseroides (prismas esféricos) cuya densidad varía en
profundidad según una función continua arbitraria.
Esta metodología aproxima los campos gravitatorios mediante una \ac{GLQ} junto
con la aplicación de dos algoritmos de discretización que controlan
automáticamente la precisión de la aproximación dividiendo adaptativamente el
tesseroide original en otros más pequeños.
El primero es un algoritmo preexistente de discretización adaptativa
bidimensional que reduce el error debido a la distancia entre el tesseroide
y el punto de cómputo.
El segundo es un nuevo algoritmo de discretización basado en la densidad que
reduce los errores debido a la variación de la densidad con la profundidad.
La cantidad de subdivisiones que realiza cada algoritmo es indirectamente
controlada por dos parámetros: el \emph{ratio distancia-tamaño} y el
\emph{ratio delta}.
Hemos obtenido soluciones analíticas de los campos gravitatorios generados por
un cascarón esférico con densidades variables a lo largo de la dirección radial
y los hemos comparado con los resultados del modelo numérico para densidades
lineales, exponenciales y sinusoidales.
Las densidades oscilantes fueron utilizadas con la única intención de someter
al algoritmo a sus límites y no para emular un escenario real.
Estas comparaciones nos han permitido obtener valores óptimos para los ratio
distancia-tamaño y delta que garantizan una precisión de 0.1\% en relación con
las soluciones analíticas.
Los valores óptimos del ratio distancia-tamaño para el potencial gravitatorio
y su gradiente son 1 y 2.5, respectivamente.
La discretización basada en la densidad no produce subdivisiones en el caso de
densidad lineal, pero un ratio delta de 0.1 es necesario para densidades
exponenciales y para la mayoría de las sinusoidales.
Estos valores pueden ser extrapolados para cubrir los usos más comunes, los
cuales son siempre más sencillos que perfiles de densidad que oscilan.
Sin embargo, los ratios densidad-tamaño y delta pueden ser configurados por el
usuario para aumentar la precisión de los resultados a expensas de un aumento
en el tiempo de cómputo.
Por último, hemos aplicado esta nueva metodología para modelar la Cuenca
Neuquina, una cuenca de antepaís en Argentina, con una profundidad máxima de
5000~m y utilizando una densidad exponencial.


\section{Introducción}

La variación de la densidad de la litósfera con la profundidad ha sido
estudiada por casi un siglo.
A lo largo de este tiempo, varias relaciones entre la densidad y la profundidad
han sido propuestas para diferentes tipos de rocas
\citep[por ejemplo~][]{maxant1980, rao1986, rao1993, rao1994}.
Además, densidades que varían con la profundidad han sido utilizadas en el
modelado directo e inverso de datos gravitatorios, principalmente aplicados
a cuencas sedimentarias
\citep{cordell1973, rao1986, cowie1990, rao1993, rao1994, zhang2001,
welford2010}.
Estos modelos directos han sido desarrollados para cuerpos bidimensionales
o tridimensionales en coordenadas Cartesianas, lo que limita su aplicación
a escalas locales.
La llegada de la gravimetría satelital ha proveído mediciones del campo
gravitatorio terrestre con cobertura global, permitiendo el modelado
e interpretación en escalas regionales o globales.
Por esta razón, diseñar métodos de modelado directo que reproducen las
anomalías de gravedad para esas escalas resulta muy importante.

Con el objetivo de tomar en consideración la curvatura de la Tierra, muchos
modelos directos globales se definen en coordenadas esféricas geocéntricas
(ver Capítulo~\ref{cha:fundamentos}).
Un abordaje común es discretizar la Tierra en tesseroides \citep{anderson1976},
es decir prismas esféricos, los cuales son definidos por el volumen que
delimitan pares de latitud, longitud y radios (Fig.~\ref{fig:tesseroid}).
Los campos gravitatorios generado por un tesseroide en cualquier punto exterior
vienen dados por integrales de volumen que deben ser aproximadas numéricamente.
La literatura ofrece dos abordajes principales: uno involucra expansiones en
series de Taylor \citep{heck2006, grombein2013}, mientras que la otra hace uso
de la \ac{GLQ}
La expansión en series de Taylor no resulta adecuada para desarrollar un
algoritmo para densidades variables según funciones arbitrarias.
Los diferentes términos de las series deberían obtenerse para cada una de las
posibles funciones de densidad.
Por el contrario, una función de densidad arbitraria puede incluirse dentro de
la \ac{GLQ} sin necesidad de modificar el método de integración.
Es por esta razón que haremos foco en los métodos basados en la \ac{GLQ} de
aquí en adelante.

El principal desafío de la integración por \ac{GLQ} es la pérdida de precisión
que ocurre cuando el punto de cómputo se acerca al tesseroide \citep{ku1977}.
\citet{uieda2016} desarrollaron un método a partir del algoritmo de
discretización adaptativa tridimensional de \citet{li2011} para obtener de
manera automática campos gravitatorios de tesseroides con densidad uniforme con
0.1\% de precisión.
El algoritmo divide recursivamente un tesseroide en otros más pequeños cuando
un determinado umbral es excedido, esto es cuando la distancia normalizada
entre el punto de cómputo es mayor que un parámetro denominado \emph{ratio
distancia-tamaño} ($D$).
\citet{uieda2016} obtuvieron además valores estándar de $D$ para el cómputo del
potencial gravitatorio, las componentes de su gradiente y del tensor de Marussi
comparando los resultados de la integración numérica con los campos generados
por un cascarón esférico.

Dos publicaciones recientes presentan abordajes alternativos para calcular los
campos gravitatorios de tesseroides homogéneos e incorporan metodologías para
el caso de tesseroides con densidades variables en profundidad.
\citet{fukushima2018} ha resuelto analíticamente la integral correspondiente al
potencial gravitatorio en la dirección radial, obteniendo una integral de
superficie, la cual es posteriormente resuelta dividiendo condicionalmente el
tesseroide y aplicando la cuadratura exponencial doble.
El gradiente del potencial y las componentes del tensor de Marussi son
calculadas posteriormente por diferencias finitas.
\citet{fukushima2018} además generaliza el método para tesseroides con una
densidad que varía con el radio según una función polinomial de grado
arbitrario.
\citet{lin2019} han comparado las diferentes metodologías de integración
y discretización para tesseroides homogéneos.
A partir de este análisis han desarrollado un método combinado:
Para puntos de cómputo cercanos al tesseroide, utilizan una integración
\ac{GLQ} junto con una discretización adaptativa basada en \citet{uieda2016}
pero solo aplicada a las dimensiones horizontales.
Si el punto de cómputo se encuentra más allá de una cierta distancia de
truncamiento, aplican una aproximación en serie de Taylor de segundo orden,
junto con la subdivisión desarrollada por \citet{grombein2013}.
\citet{lin2019} además introducen una variación de su método combinado para
calcular los campos gravitatorios generados por tesseroides con densidades
lineales en la dirección radial.

Los desarrollos de \citet{lin2019} y \citet{fukushima2018} se limitan
a funciones de densidad polinomiales.
Si bien la mayoría de las funciones suaves pueden aproximarse por funciones
lineales de a pasos, la elección del intervalo de discretización no es ni
directa ni automática para el caso general.
Aunque existen muchos algoritmos de aproximación lineales de a pasos
automatizan este proceso \citep{ketkov1969, vandewalle1975, imamoto2008,
ahmadi2013}, requieren un número fijo de intervalos de discretización o están
diseñados para ser utilizados solo en funciones convexas.
El uso de estos algoritmos limitaría el dominio de funciones de densidades que
pueden ser asignadas a tesseroides y no garantizarían un proceso completamente
automático.
Además, es bien conocido que el uso de polinomios de alto grado para aproximar
una función altamente variable produce resultados inestables cuando se
extrapola más allá del dominio de datos.
Estos obstáculos podrían hacer que aproximar densidades mediante aproximaciones
lineales de a pasos o por polinomios de alto grado no resulte adecuado para
inversiones de gravedad no lineales \citep[e.g.][]{uieda2017} si la función
densidad es altamente variable en profundidad.

Presentamos un nuevo algoritmo para el cálculo del potencial gravitatorio y su
gradiente generado por un tesseroide con una función continua de densidad sobre
cualquier punto de cómputo externo.
Esta metodología está basada en una integración \ac{GLQ} tridimensional, una
versión bidimensional del algoritmo de discretización adaptativa de
\citet{uieda2016} \citep[de acuerdo con~][]{lin2019}, y un nuevo algoritmo
de discretización basado en la densidad.
Para garantizar la precisión de la aproximación numérica, hemos determinado
empíricamente valores óptimos para los parámetros que controlan las
discretizaciones, comparando los resultados numéricos con las soluciones
analíticas de cascarones esféricos.
Finalmente, hemos aplicado la nueva metodología para modelar la Cuenca
Neuquina, Argentina, utilizando tesseroides con densidades lineales
y exponenciales en profundidad.

%%%%%%%%%%%%%%%%%%%%%%%%%%%%%%%%%%%%%%%%%%%%%%%%%%%%%%%%%%%%%%%%%%%%%%%%%%%%%%

\section{Metodología}

Consideremos un tesseroide en un sistema de coordenadas esféricas geocéntricas
definido por pares de latitudes ($\phi_1$, $\phi_2$), longitudes ($\lambda_1$,
$\lambda_2$) y radios ($r_1$, $r_2$).
Definimos un punto de computo externo $\mathbf{p}$ localizado en un radio $r$,
una latitud $\phi$ y una longitud $\lambda$.
\citet{grombein2013} proveen formulaciones eficientes para las integrales de
volumen del potencial gravitatorio generado por un tesseroide de densidad
homogénea, junto con las componentes de su gradiente
(ecs.~\ref{eq:potencial-tesseroide}, \ref{eq:gx-tesseroide},
\ref{eq:gy-tesseroide}, \ref{eq:gz-tesseroide}).
Aquí vamos a considerar los casos en los cuales el tesseroide posee una
densidad variable con respecto a la coordenada radial $r$ según una función
continua arbitraria $\rho(r)$.
Por lo tanto, las integrales del potencial gravitatorio y las componentes de su
gradiente se ven ligeramente modificadas:

\begin{equation}
    V(r,\phi,\lambda) = G
    \int\limits_{\lambda_1}^{\lambda_2}
    \int\limits_{\phi_1}^{\phi_2}
    \int\limits_{r_1}^{r_2}
    \frac{\rho(r')}{\ell} \kappa \,  dr' d\phi' d\lambda',
\label{eq:tesseroid-pot}
\end{equation}

\begin{equation}
    g_{\alpha}(r,\phi,\lambda) = G
    \int\limits_{\lambda_1}^{\lambda_2}
    \int\limits_{\phi_1}^{\phi_2}
    \int\limits_{r_1}^{r_2}
    \rho(r') \frac{\Delta\alpha}{\ell^3}
    \kappa \, dr' d\phi' d\lambda',
\label{eq:tesseroid-grav}
\end{equation}

\noindent donde $\alpha \in \{x, y, z\}$, y $\Delta \alpha$ vienen dados por
las ecuaciones~\ref{eq:delta-x}, \ref{eq:delta-y} y \ref{eq:delta-z}.

\subsection{Integración por Cuadratura de Gauss-Legendre}

Aplicando una \ac{GLQ} de orden $N$ podemos aproximar a cada integral definida
por las ecuaciones~\ref{eq:tesseroid-pot} and~\ref{eq:tesseroid-grav} por una
suma ponderada de los kernel de integración evaluados en las raíces de un
polinomio de orden $N$ \citep[p.~390]{hildebrand1987}, conocidos como los nodos
de la cuadratura.
A diferencia de los tesseroides con densidad constante, la función de densidad
$\rho(r)$ debe ser incluida dentro del integrando y evaluada en los nodos de la
cuadratura:

\begin{equation}
        \int\limits_{\lambda_1}^{\lambda_2}
        \int\limits_{\phi_1}^{\phi_2}
        \int\limits_{r_1}^{r_2}
        \rho(r') f(r', \phi', \lambda')
        dr' d\phi' d\lambda' \approx
        A
        \sum\limits_{i=1}^{N^r}
        \sum\limits_{j=1}^{N^\phi}
        \sum\limits_{k=1}^{N^\lambda}
        W_i^r W_j^\phi W_k^\lambda
        \rho(r_i) f(r_i, \phi_j, \lambda_k),
\label{eq:glq-var-dens}
\end{equation}

\noindent donde $A$ es una constante definida en la
ecuación~\ref{eq:glq-resize-factor}, $f(r', \phi', \lambda')$ es el kernel
correspondiente a un tesseroide con densidad homogénea \citep{grombein2013},
$(r_i, \phi_j, \lambda_k)$ son las coordenadas de los nodos de la cuadratura,
$N^r$, $N^\phi$, $M^\lambda$ son los órdenes de la cuadratura y $W_i^r$,
$W_j^\phi$, $W_k^\lambda$ son los pesos ponderados en la dirección radial,
latitudinal y longitudinal, respectivamente.
Vale la pena notar que aplicar la \ac{GLQ} es equivalente a aproximar el
tesseroide por $N^r \times N^\phi \times N^\lambda$ masas puntuales localizadas
en los nodos de la cuadratura. \citep{ku1977, asgharzadeh2007}.


\subsection{Discretización adaptativa bidimensional}

\citet{ku1977} dio cuenta de que la integración por \ac{GLQ} se vuelve menos
precisa a medida que el punto de cómputo se acerca al tesseroide.
Una manera de prevenir que esto suceda sería incrementar el orden de la
cuadratura.
Al hacer esto, incrementaríamos uniformemente la cantidad de masas puntuales
utilizadas para aproximar el tesseroide.
Sin embargo, sólo es necesario incrementar la concentración de masas puntuales
en cercanías del punto de cómputo \citep{uieda2016}.
Alternativamente,  \citet{li2011} han propuesto un algoritmo de discretización
adaptativa que mantiene fijo el orden de la cuadratura y divide el tesseroide
según una relación entre la distancia al punto de cómputo y las dimensiones
del tesseroide.
Este algoritmo produce un cómputo más eficiente, ya que produce un aumento en
la concentración de masas puntuales sólo en las regiones donde es más
necesario. \citet{uieda2016} han desarrollado una versión modificada de este
algoritmo junto con una implementación computacional eficiente.
Ambas versiones del algoritmo subdividen el tesseroide en las direcciones
latitudinal, longitudinal y radial, por ende podemos definirlos como
\emph{algoritmos de discretización adaptativa tridimensionales}.
Por otro lado, \citet{lin2019} han propuesto un \emph{algoritmo de
discretización adaptativa bidimensional} que subdivide al tesseroide solo en
las direcciones latitudinal y longitudinal.
Remover una dimensión de la discretización produce un cómputo más eficiente ya
que reduce la cantidad de subdivisiones en el modelo, aunque manteniendo una
precisión aceptable \citep{lin2019}.

A lo largo de este capítulo nos guiaremos con los resultados de \citet{lin2019}
y haremos uso de una versión bidimensional del algoritmo de discretización
adaptativa de \citet{uieda2016}.
Lo que sigue a continuación es un resumen del algoritmo. El lector o la lectora
puede referirse a \citet{uieda2016} para una descripción más detallada.

\textit{Paso 1}: Corroboramos que el tesseroide satisface la siguiente
desigualdad para sus dimensiones longitudinales y latitudinales
$L_i\ (i \in \{\lambda, \phi\})$:

\begin{equation}
    \frac{d}{L_i} \geq D,
    \label{eq:condition}
\end{equation}

\noindent
donde $D$ es un escalar positivo denominado \emph{ratio de distancia-tamaño},
$d$ es la distancia entre el punto de cómputo y el centro geométrico del
tesseroide:

\begin{equation}
    d = \left[
        r^2 + r_t^2 - 2 r r_t \cos\psi_t
        \right]^{\frac{1}{2}} ,
    \label{eq:distance}
\end{equation}

\begin{equation}
    \cos\psi_t =
        \sin\phi\sin\phi_t + \cos\phi\cos\phi_t\cos(\lambda - \lambda_t),
\end{equation}

\begin{equation}
    r_t = \frac{r_2 + r_1}{2}, \quad
    \phi_t = \frac{\phi_2 + \phi_1}{2}, \quad
    \lambda_t = \frac{\lambda_2 + \lambda_1}{2},
\end{equation}

\noindent
y las dimensiones del tesseroide se definen como:

\begin{equation}
    L_\lambda = r_2 \arccos(\sin^2\phi_t +
        \cos^2\phi_t\cos(\lambda_2 - \lambda_1)),
    \label{eq:sizelon}
\end{equation}

\noindent y

\begin{equation}
    L_\phi = r_2 \arccos(\sin\phi_2\sin\phi_1 + \cos\phi_2\cos\phi_1).
\end{equation}

\textit{Paso 2}:
Si todas de las dimensiones del tesseroide satisfacen la desigualdad~\ref{eq:condition},
entonces calculamos el efecto gravitatorio del tesseroide usando una \ac{GLQ}
de segundo orden (Eq.~\ref{eq:glq-var-dens}).
Agregamos el efecto a un total acumulado.

\textit{Paso 3}:
Si la desigualdad~\ref{eq:condition} no es satisfecha por una o ambas
dimensiones (longitudinal o latitudinal), dividimos el tesseroide al medio a lo
largo de esa dimensión o dimensiones.
Repetimos los pasos 1~a~3 para todos los tesseroides más pequeños hasta que
ninguno de ellos viole la desigualdad~\ref{eq:condition}.

\textit{Último paso}:
Al finalizar el algoritmo, una \ac{GLQ} de segundo orden habrá sido aplicada
a cada tesseroide más pequeño y el total acumulado del paso 2 será equivalente
al efecto gravitatorio del tesseroide original.

El ratio densidad-tamaño $D$ determina cuántas veces el tesseroide será
dividido.
Por lo tanto, su valor regula tanto la precisión del algoritmo como su tiempo
de cómputo.
Un valor óptimo de $D$ no puede ser calculado directamente a partir del nivel
de precisión deseado.
En cambio, es posible determinarlo empíricamente comparando los resultados
numéricos con la solución analítica para un cascarón esférico.
\citet{uieda2016} han utilizado un cascarón de densidad homogénea para
determinar los valores óptimos de $D$ para su algoritmo aplicado a tesseroides
de densidad constante.
Aquí repetiremos dicho experimento numérico utilizando las expresiones
analíticas de los campos gravitatorios generados por cascarones con densidad
variable según funciones continuas de $r$.
Estos experimentos también corroborarán si los mismos valores de $D$
determinados por \citet{uieda2016} pueden ser utilizados en conjunto con un
algoritmo de discretización adaptativa bidimensional.


\subsection{Algoritmo de discretización basado en densidad}

La integración numérica considerando una función continua de densidad introduce
un nuevo tipo de problema: el error de integración que surge de utilizar solo
unos pocos nodos de cuadratura para considerar la variación de la densidad del
tesseroide.
El algoritmo de discretización adaptativa tridimensional puede ayudar a reducir
este tipo de error agregando más masas puntuales en la dirección radial.
Sin embargo, este no tiene en cuenta la función de densidad en sí misma a la
hora de decidir cómo subdividir al tesseroide, y por lo tanto no es el método
más adecuado para realizar esta tarea.

En este trabajo hemos desarrollado un algoritmo de discretización
complementario que subdivide al tesseroide en la dirección radial tomando en
consideración las variaciones de la función densidad.
Este \emph{algoritmo de discretización basado en la densidad} se aplica antes
del \emph{algoritmo de discretización adaptativa bidimensional} descripto en la
sección anterior.
En resumen, el algoritmo divide al tesseroide a lo largo de la dirección radial
en las profundidades a las cuales ocurre la "máxima variación de densidad".

Consideremos un tesseroide, que llamaremos \emph{original}, con una función de
densidad dada por $\rho(r')$.
Antes de comenzar a aplicar el algoritmo de discretización basado en la
densidad, normalizamos la función densidad al intervalo $[0, 1]$ de la
siguiente manera:

\begin{equation}
    \rho_n(r') =
    \frac{\rho(r') - \rho_\text{min}}{\rho_\text{max} - \rho_\text{min}},
\end{equation}

\noindent donde $\rho_\text{min}$ y $\rho_\text{max}$ son el mínimo y el máximo
valor de densidad dentro de los límites de tesseroide.
Hacemos hincapié en que esta función densidad normalizada no se verá modificada
a lo largo del algoritmo.
En caso de que la función densidad sea constante, los mínimos y máximos serán
iguales y por ende el algoritmo de discretización basado en densidad no será
aplicado.

\begin{figure}
\centering
\includegraphics[width=\linewidth]
    {figs/tesseroids-variable-density/density-based-discretization-algorithm.pdf}
\caption{
    Ejemplo de la aplicación del algoritmo de discretización basado en densidad
    a una función densidad no lineal.
    (a)~La función densidad normalizada $\rho_n(r')$ (azul), los límites
    actuales del tesseroide (puntos naranja), y la \emph{línea recta}
    $\rho_l(r')$ (línea naranja).
    Las líneas de a trazos rojas representan la máxima diferencia de densidad
    $\Delta \rho (r')$ a la cual el tesseroide será subdividido (asumiendo que
    la desigualdad~\ref{eq:delta-density} no se satisface).
    (b)~Segunda iteración del algoritmo con una nueva \emph{línea recta} y máxima
    diferencia de densidad. El tesseroide será dividido a la profundidad
    indicada por la linea roja de a trazos.
    (c)~Tercera iteración del algoritmo.
    (d)~ Salida final del algoritmo de discretización basado en densidad,
    asumiendo que los cuatro tesseroides nuevos satisfacen la
    desigualdad~\ref{eq:delta-density}.
}
\label{fig:density-discretization-algorithm}
\end{figure}


El algoritmo puede comprenderse a través de los siguientes pasos
(Fig.~\ref{fig:density-discretization-algorithm}):

\textit{Paso 1}:
Definimos una \emph{línea recta} $\rho_l(r')$ que toma los mismos valores que
la función densidad normalizada $\rho_n(r')$ en los extremos del tesseroide
($r_1$ and $r_2$):

\begin{equation}
    \rho_l(r') =
    \frac{ \rho_n(r_2) - \rho_n(r_1) }{ r_2 - r_1 } (r' - r_1) + \rho_n(r_1).
    \label{eq:density-reference-line}
\end{equation}

\textit{Paso 2}:
Evaluamos la función de densidad normalizada y la \emph{línea recta} en un
rango de $N$ radios entre $r_1$ y $r_2$. Hemos optado por un $N = 101$, pero el
valor específico de $N$ no es crítico para el funcionamiento del algoritmo.

\textit{Paso 3}:
Calculamos la diferencia absoluta entre los valores de la \emph{línea recta}
y la función densidad normalizada:

\begin{equation}
    \Delta \rho (r') = | \rho_n(r') - \rho_l(r') |.
    \label{eq:density-abs-diff}
\end{equation}

\textit{Paso 4}:
Si la siguiente desigualdad se satisface, el tesseroide no será subdividido:

\begin{equation}
    \text{max}\{ \Delta \rho(r') \} \frac{L_r}{L_r^\text{orig}} \le \delta,
    \label{eq:delta-density}
\end{equation}

\noindent
donde $L_r$ es la dimensión radial del tesseroide considerado para ser
subdividido,

\begin{equation}
    L_r = r_2 - r_1,
\end{equation}

\noindent $L_r^\text{orig}$ es la dimensión radial del tesseroide
\emph{original}, y $\delta$ es una constante positiva que denominaremos
\emph{ratio delta}.

\textit{Paso 5}:
Si la desigualdad~\ref{eq:delta-density} no se satisface, entonces dividimos el
tesseroide en dos a una profundidad dada por el radio $r_\text{max}$ a la cual
la máxima diferencia absoluta (Eq.~\ref{eq:density-abs-diff}) tiene lugar.
Repetimos los pasos 1~a~5 para cada tesseroide pequeño producido por este paso.

\textit{Paso final}:
Una vez que todos los pequeños tesseroides satisfacen la
desigualdad~\ref{eq:delta-density}, cada uno es sometido al algoritmo de
discretización adaptativa bidimensional descripto anteriormente con el objeto
de calcular sus efectos gravitatorios.

En la primer iteración, la relación $L_r/L_r^\text{orig}$ es igual a 1, ya que
el tesseroide que se considera para subdividir corresponde al \emph{original}.
En las subsiguientes iteraciones, esta relación sera progresivamente menor que
1 a medida que los tesseroides son cada vez más pequeños.
La intención de este comportamiento es limitar la cantidad de divisiones a un
numero que reduzca significativamente el error numérico:
dividir un tesseroide grande con un $\text{max}\{ \Delta \rho(r') \}$ bajo
mejoraría la precisión de la integración en mayor medida que dividiendo un
tesseroide pequeño con un $\text{max}\{ \Delta \rho(r') \}$ mayor.

A mayor valor de $\delta$, se llevarán a cabo menor cantidad de divisiones,
y viceversa.
Por ende, el valor de $\delta$ controla cuántas veces los tesseroides serán
divididos basados en la función densidad, e indirectamente, determina la
precisión y el tiempo de cómputo de la integración numérica.
Esto hace surgir la necesidad de determinar un valor máximo de $\delta$ que
garantice una precisión aceptable mientras minimiza el tiempo de cómputo.


\subsection{Resumen del algoritmo}

En resumen, dado un tesseroide con densidad variable en profundidad segun una
función continua arbitraria, proponemos seguir los siguientes pasos para
calcular numéricamente sus campos gravitatorios en cualquier punto exterior:

\textit{Paso 1:}
Aplicar el \emph{algoritmo de discretización basado en densidad} para
subdividir al tesseroide a lo largo de la dirección radial, produciendo un
conjunto de tesseroides con las mismas dimensiones longitudinales
y latitudinales que el original, pero con diferentes límites radiales.

\textit{Paso 2:}
Aplicar el \emph{algoritmo de discretización adaptativa bidimensional} a cada
tesseroide obtenido en el paso anterior.
Si es necesario, el algoritmo subdividirá cada tesseroide en las direcciones
longitudinal y latitudinal, generando un conjunto de tesseroides más pequeños.

\textit{Paso 3:}
Aplicar una \ac{GLQ} de segundo orden para calcular numéricamente los campos
gravitatorios (Eq.~\ref{eq:glq-var-dens}) generados por cada tesseroide
obtenido en el paso anterior. La integración numérica incluye la función
densidad y puede ser aplicada sin modificaciones a cualquier función continua.
La suma de todos estos resultas corresponde al campo gravitatorio generado por
el tesseroide original.


\subsection{Implementación por Software}

Hemos implementado los algoritmos descriptos en las secciones anteriores
mediante el uso del lenguaje de programación Python.
El software está basado en la implementación en Python de \citet{uieda2016}
incluida en la librería Fatiando a Terra v0.5 \citep{uieda2013}.
Los pasos que involucran un mayor tiempo de cómputo han sido escritos en
lenguaje Cython para alcanzar un mejor desempeño.
Aprovechamos la naturaleza dinámica del lenguaje Python para permitir funciones
de densidades definidas por el usuario o la usuaria como entradas del software.
Por ende, nuestro código puede evaluar sin modificaciones funciones lineales,
exponenciales, polinomiales, sinusoidales, splines cúbicas o cualquier otra
función de densidad continua.
Esta implementación se encuentra disponible libremente bajo la licencia
open-source BSD 3-clause.
Todo el código fuente, los scripts de Python, los datos y resultados se
encuentran disponibles a través del siguiente repositorio
\href{https://doi.org/10.6084/m9.figshare.8239622}{doi.org/10.6084/m9.figshare.8239622}
\citep{soler2019b} or
\href{https://github.com/pinga-lab/tesseroid-variable-density}{github.com/pinga-lab/tesseroid-variable-density}.
El repositorio también contiene instrucciones para replicar los resultados
presentados en este capítulo.


%%%%%%%%%%%%%%%%%%%%%%%%%%%%%%%%%%%%%%%%%%%%%%%%%%%%%%%%%%%%%%%%%%%%%%%%%%%%%

\section{Determinación de los ratio distancia-tamaño y delta}

El ratio distancia-tamaño $D$ de la discretización adaptativa y el ratio delta
$\delta$ de la discretización basada en la densidad determinan cuantas veces
cada tesseroide será dividido y por ende controlan indirectamente el error
numérico de las integraciones.
Debemos determinar valores óptimos para $D$ y $\delta$ si deseamos asegurar
tanto una precisión numérica aceptable así como una eficiencia computacional de
los algoritmos.

\citet{uieda2016} compararon los resultados de las integraciones numéricas de
tesseroides homogéneos con las soluciones analíticas de un cascarón esférico
\citep{mikuska2006, grombein2013} con el objetivo de obtener valores
predeterminados para el ratio distancia-tamaño $D$.
Seguiremos esta idea, pero en nuestro caso el cascarón esférico debe tener el
mismo perfil de densidad que nuestro modelo de tesseroides. \citet{lin2019}
obtuvieron la solución analítica del potencial gravitatorio generado por un
cascarón esférico con densidad lineal en la coordenada radial.
Aplicando el Teorema del Cascarón de Newton \citep{chandrasekhar1995,
binney2008}, podemos derivar expresiones para el potencial gravitatorio de un
cascarón esférico con densidades lineales, exponenciales o sinusoidales (ver
Apéndice~\ref{cha:shell}).

Con el objetivo de comparar los resultados numéricos con las soluciones
analíticas debemos construir modelos de un cascarón esférico a partir de
tesseroides.
Dividimos entonces el cascaron esférico a lo largo de las direcciones
longitudinales y latitudinales obteniendo un modelo de cascarón conformado por
$6 \times 12 = 72$ tesseroides de un tamaño de $30^\circ \times 30^\circ$.
Hemos definido varios modelos de cascarones con diferentes espesores
(Tabla~\ref{tab:shell-models}) para poder evaluar la discretización basada en
densidad en diferentes escenarios.
Los valores de espesor fueron elegidos para cubrir un amplio rango de
posibles aplicaciones: desde modelos topográficos a escalas litosféricas.
Dado que la cantidad de subdivisiones en la discretización adaptativa será
proporcional al tamaño de los tesseroides (ec.~\ref{eq:condition}),
algunas de estas configuraciones representan los peores casos.
La mayoría de las aplicaciones prácticas usarían tesseroides más pequeños que
$30^\circ \times 30^\circ \times 1000\ \text{km}$.

Las diferencias entre la solución analítica y los resultados numéricos pueden
ser calculados en un único punto de cómputo debido a la simetría rotacional del
cascarón esférico.
Sin embargo, los resultados numéricos dependen de la posición relativa entre el
punto de cómputo y el tesseroide \citep{ku1977, asgharzadeh2007, uieda2016}.
Tendremos en cuenta este fenómeno calculando dichas diferencias sobre grillas
regulares y almacenando únicamente la diferencia máxima absoluta.
Estos cálculos serán realizados sobre cuatro grillas diferentes
(Tabla~\ref{tab:grids}):
una grilla local sobre el polo, una grilla local en el ecuador, una grilla
global a altitud cero, y una grilla global a una altitud de 260km sobre la
superficie del cascarón (representando la altitud nominal del satélite GOCE).
Estas grillas cubren un amplio espectro de escenarios y asegurarán una
precisión aceptable para cada uno de ellos.

Las comparaciones entre las soluciones analíticas y los resultados numéricos
serán llevadas a cabo utilizando funciones de densidades lineales,
exponenciales y sinusoidales.
La densidad sinusoidal es incluida con el objetivo de evaluar la aproximación
numérica en sus límites de desempeño.
Repetimos estas comparaciones para cada combinación de modelos de tesseroides
en la Tabla~\ref{tab:shell-models} y grillas de la Tabla~\ref{tab:grids}.
A partir de estos resultados generalizaremos valores óptimos para $D$
y $\delta$ que garantizan errores numéricos menores a 0.1\% en comparación con
las soluciones analíticas del cascarón esférico en la mayoría de los casos.

\begin{table}
\caption{
    Descripción de los modelos de tesseroides utilizados para construir
    cascarones esféricos y caracterizar la precisión de las integraciones
    numéricas.
    El radio exterior ($R_2$) de cada modelo es igual al radio medio de la
    Tierra (6378.137 km), mientras que le radio interno ($R_1$) es determinado
    por el espesor del cascarón.
    La cantidad total y las dimensiones horizontales de los tesseroides en cada
    modelo de cascarón están detalladas según dimensiones latitudinales
    y longitudinales, respectivamente.
    \newline
}
\label{tab:shell-models}
\centering
\begin{tabular}{rccccc}
    Espesor & Tamaño de cada tesseroide  & Cantidad de tesseroides \\ \hline
    0.1 km  & $30^\circ \times 30^\circ$ & $6 \times 12 = 72$ \\
    1 km    & $30^\circ \times 30^\circ$ & $6 \times 12 = 72$ \\
    10 km   & $30^\circ \times 30^\circ$ & $6 \times 12 = 72$ \\
    100 km  & $30^\circ \times 30^\circ$ & $6 \times 12 = 72$ \\
    1000 km & $30^\circ \times 30^\circ$ & $6 \times 12 = 72$ \\
\end{tabular}
\end{table}

\begin{table}
\caption{
    Descripción de las grillas de putos de cómputo utilizadas para caracterizar
    la precisión de las integraciones numéricas.
    La altitud de las grillas es definida por encima del radio medio de la
    Tierra.
    \newline
}
\label{tab:grids}
\centering
\begin{tabular}{lccc}
    Nombre & Espaciado de la grilla & Región (grados) & Altitud (km)
    \\ \hline
    Polo      & $0.1^\circ$ &   0E /   1E / 89N / 90N & 0   \\
    Ecuador   & $0.1^\circ$ &   0E /   1E /  0N / 1N  & 0   \\
    Global    & $ 10^\circ$ & 180W / 180E / 90S / 90N & 0   \\
    Satélite  & $ 10^\circ$ & 180W / 180E / 90S / 90N & 260 \\
\end{tabular}
\end{table}


\subsection{Densidad Lineal}

El potencial gravitatorio y su derivada vertical que genera un cascarón
esférico con una función de densidad lineal dada por

\begin{equation}
    \rho(r') = ar' + b,
    \label{eq:density-linear}
\end{equation}

\noindent posee soluciones analíticas dadas por las
ecuaciones~\ref{eq:shell-pot-linear} y~\ref{eq:shell-gz}. Los valores de los
coeficientes $a$ y $b$ pueden ser elegidos de forma tal que la densidad asuma
valores valores iguales a  $\rho_\text{in} = 3300$~kg/m$^3$ y
$\rho_\text{out} = 2670$~kg/m$^3$ en los radios interno ($R_\text{in}$)
y externo ($R_\text{out}$) del cascarón, respectivamente:

\begin{equation}
    a = \frac{\rho_\text{out} - \rho_\text{in}}{R_\text{out} - R_\text{in}},
\end{equation}

\begin{equation}
    b = \rho_\text{out} -
    \frac{\rho_\text{out} - \rho_\text{in}}{R_\text{out} - R_\text{in}} R_\text{out}.
\end{equation}


\begin{figure}
\centering
\includegraphics[width=\linewidth]{figs/tesseroids-variable-density/linear-density-diffs.pdf}
\caption{
    Diferencias entre los campos gravitatorios generados por cada modelo de
    cascarón hecho con tesseroides y sus respectivas soluciones analíticas en
    función del ratio distancia-tamaño $D$.
    Cada modelo posee una densidad lineal (ec.~\ref{eq:density-linear}).
    Los cálculos fueron realizados sobre las cuatro grillas descriptas en la
    Tabla~\ref{tab:grids}.
    Cada curva representa la diferencia máxima absoluta entre los resultados
    numéricos y las solución analítica para un dado modelo de cascarón.
    Debido a la linealidad de la función densidad, el algoritmo de
    discretización basado en densidad no es aplicado.
    Las diferencias están reportadas como un porcentaje de las soluciones
    analíticas.
    Las líneas horizontales de a trazos y color negro representan el umbral de
    precisión de 0.1\%.
}
\label{fig:D-linear}
\end{figure}

Las diferencias absolutas definidas en la ecuación~\ref{eq:density-abs-diff}
son siempre cero para cualquier tipo de densidad lineal.
Como resultado, la desigualdad~\ref{eq:delta-density} será siempre satisfecha
y no será necesario subdividir el tesseroide en la dirección radial durante la
discretización basada en densidad.
Por lo tanto, solo la discretización adaptativa bidimensional es el único
mecanismo que controla la precisión de la integración numérica para el caso de
densidad lineal.
Por esta razón ignoraremos los valores de $\delta$ y solo determinaremos el
mínimo valor de $D$ necesario para garantizar una precisión aceptable.

Calculamos el potencial gravitatorio ($V$) y su derivada vertical ($g_z$)
para cada modelo de cascarón esférico definido en la
Tabla~\ref{tab:shell-models} en cada grilla de cómputo de la
Tabla~\ref{tab:grids}.
Las derivadas horizontales del potencial son iguales a cero fuera del cascarón
debido a la simetría rotacional, y por ende son omitidos del análisis.
Estos cálculos son repetidos para valores de ratio distancia-tamaño $D$ desde
0.5 a 5 con incrementos de 0.5.
Luego calculamos la diferencia absoluta entre los resultados numéricos y las
soluciones analíticas para el cascarón.
La Figura~\ref{fig:D-linear} muestra la máxima diferencia absoluta para cada
modelo de cascarón y grilla de cómputo como función de $D$.
Las diferencias se muestran relativas a la solución analítica para cada
cascarón.
Finalmente, seleccionamos el valor óptimo de $D$ como el menor valor para el
cual la diferencia es menor a 0.1\%.

A partir de la Figura~\ref{fig:D-linear} podemos observar que los errores
relativos para el potencial $V$ y para su derivada vertical $g_z$ caen por
debajo del umbral de 0.1\% para valores de $D=1$ y $D=2.5$, respectivamente.
Notablemente, un valor de $D=2$ sería suficiente para $g_z$ en el caso de una
grilla a altitud satelital.
Para el resto de las configuraciones, dichos valores son consistentes
e independientes del espesor de cascarón o de la ubicación geográfica.


\subsection{Densidad exponencial}

Para una función de densidad exponencial, la discretización basada en densidad
será aplicada con anterioridad al algoritmo de discretización adaptativa.
Esto significa que los valores óptimos para el ratio distancia-tamaño $D$ y el
ratio delta $\delta$ (ec.~\ref{eq:delta-density}) deben ser determinados
simultáneamente.
Para ello llevamos a cabo un análisis de error similar al que aplicamos en el
caso de densidad lineal.
Ahora el cascarón esférico poseerá una densidad según una función exponencial
que asume los valores de
$\rho_\text{out} = 2670$~kg/m$^3$ y $\rho_\text{in} = 3300$~kg/m$^3$
en las superficies externas e internas, respectivamente, y que se define de la
siguiente manera:

\begin{equation}
    \rho(r') = A e^{- b \frac{r' - R_1}{R_2 - R_1}} + C,
    \label{eq:density-exp}
\end{equation}

\noindent donde

\begin{equation}
    A = \frac{\rho_\text{in} - \rho_\text{out}}{1 - e^{-b}},
\end{equation}

\begin{equation}
    C = \rho_\text{in} - A,
\end{equation}

\noindent y $b$ es una constante adimensional que determina el grado de
variabilidad de la función. Un valor mayor de $b$ aumenta la máxima pendiente
de la función densidad (Fig.~\ref{fig:exp-densities}).

\begin{figure}
\centering
\includegraphics[width=0.5\linewidth]{figs/tesseroids-variable-density/exponential-densities.pdf}
\caption{
    Funciones de densidad exponencial asignadas a los modelos de cascarón
    esférico para la determinación del ratio $\delta$.
    Cada función densidad corresponde a diferentes valores de $b$ en la
    ecuación~\ref{eq:density-exp}.
}
\label{fig:exp-densities}
\end{figure}


\subsubsection{Exploración del espacio $D$-$\delta$}

Nuestro obejtivo es hallar una combinmación de valores para $D$ y $\delta$ que
produzcan un error numérico menor al umbral de 0.1\% mientras minimicen el
tiempo de cómputo.
Para ello hacemos uso del método de búsqueda en grilla (\emph{grid search} en
inglés) y calculamos el error numérico para cada par ($D$, $\delta$)
perteneciente a una grilla en el espacio $D$-$\delta$
(Fig.~\ref{fig:grid-search}).
Con el objeto de obtener un óptimo desempeño del algoritmo, buscamos el ($D$,
$\delta$) que minimice el número de subdivisiones de tesseroides mientras
mantengan el error numérico por debajo del umbral de 0.1\%.
Este requisito se traduce en minimizar el valor de $D$ y maximizar el valor de
$\delta$.

Dado que la búsqueda en grilla es un proceso computacionalmente costoso,
limitamos el análisis a un valor alto de $b=30$ (Fig.\ref{fig:exp-densities})
y la grilla de cómputo \emph{Global} definida en la Tabla~\ref{tab:grids}.
Calculamos la diferencia relativa entre el resultado numérico y la solución
analítica del potencial gravitatorio ($V$) y su derivada vertical ($g_z$) para
cada modelo de cascarón definido en la Tabla~\ref{tab:shell-models}.
La Figura~\ref{fig:grid-search} muestra la máxima diferencia obtenida para
todos los modelos de cascarón.
Las líneas de a trazos representan un contorno correspondiente al 0.1\% de
error relativo, por ende los puntos interiores a ella son los que caen debajo
de dicho umbral.
Además, la figura destaca los valores de $D$ obtenidos en la sección anterior
para el caso de densidad lineal ($D_\text{linear}$).

Los valores más chicos de $D$ que se encuentran debajo del umbral de 0.1\%
coinciden con los valores $D_\text{linear}$ tanto para $V$ como para $g_z$.
Estos resultados indican que los valores de $D_\text{linear}$ obtenidos pueden
ser extrapolados a los casos de densidades no lineales.
Esto no es una sorpresa, ya que la discretización adaptativa bidimensional y la
discretización basada en densidad son independientes una de la otra.
La primera divide al tesseroide en las direcciones horizontales, mientras que
la otra lo hace solo en la dirección radial.
Por ende, es de esperar que el valor óptimo de $\delta$ sea independiente del
valor óptimo de $D$.
Dado que la búsqueda en grilla fue limitada a una grilla de cómputo específica
y a un único valor de $b$, en la próxima sección realizaremos un análisis más
detallado para determinar un valor óptimo de $\delta$ para el caso de densidad
exponencial.

\begin{figure}
\centering
\includegraphics[width=0.5\linewidth]
    {figs/tesseroids-variable-density/grid-search.pdf}
\caption{
    Exploración del error numérico en el espacio $D$-$\delta$.
    Los valores porcentuales de diferencia fueron obtenidos a partir de la
    comparación entre las soluciones analíticas y las aproximaciones numéricas
    de los campos gravitatorios ($V$ and $g_z$) generados por un cascarón
    esférico con una densidad exponencial
    (ec.~\ref{eq:density-exp}) con $b=30$.
    Estas comparaciones fueron realizadas sobre la grilla de cómputo \emph{Global} (Tabla~\ref{tab:grids}),
    con los modelos de cascarón esféricos detallados en la Tabla~\ref{tab:shell-models},
    Los valores de diferencias porcentuales fueron obtenidos como la máxima
    diferencia de entre todos los modelos de cascarón.
    Los puntos interiores a la región delimitada por las lineas de a trazos son
    los que presentan un error por debajo del umbral de 0.1\%.
    Los valores de $D$ obtenidos para el caso de densidad lineal para cada
    campo gravitatorio se encuentran identificados con $D_\text{linear}$.
    }
\label{fig:grid-search}
\end{figure}


\subsubsection{Determinación del ratio delta}

Habiendo fijado valores de $D$ iguales a los obtenidos para el caso de
densidad lineal, nos encontramos en condiciones de explorar los errores de
integración como función de $\delta$ en mayor detalle.
Calcularemos la difrencia entre el error numérico y las soluciones analítica
para todas las combinaciones de grillas de cómputo (Tabla~\ref{tab:grids})
y modelos de cascarón esférico (Tabla~\ref{tab:shell-models}),
variando los valores de $\delta$ entre $10^{-3}$ y $10^{1}$.
Los cálculos son repetidos para cada valor de $b \in \{1, 2, 5, 10, 30, 100\}$
(Fig.~\ref{fig:exp-densities}) con el objeto de analizar la precisión del
método para diversos tipos de funciones exponenciales.
Dado que valores mayores de $\delta$ producen menores subdivisiones, nuestra
intención es hallar los valores máximos de $\delta$ que producen una diferencia
relativa por debajo del umbral de 0.1\%.
La Figura~\ref{fig:delta-exponential} muestra las diferencias relativas
resultantes para $V$ y $g_z$ como función de $\delta$.
Por brevedad, cada curva corresponde a la diferencia máxima entre todos los
modelos de cascarón esférico.
Las curvas correspondientes a $b=1$ y $b=2$ se encuentran por debajo del umbral
0.1\% para todos los valores de $\delta$ y no se ven modificadas para valores
  de $\delta > 0.8$, indicando que esas funciones de densidad son los
  suficientemente suaves como para poder prescindir de discretizaciones en la
  dirección radial.
Para todos los otros valores de $b$, las diferencias caen debajo del umbral si
$\delta=0.1$.
Estos resultados indican que no hay una relación significativa entre la
suavidad de la densidad exponencial y el error numérico.

\begin{figure}
\centering
\includegraphics[width=\linewidth]{figs/tesseroids-variable-density/exponential-density-diffs.pdf}
\caption{
    Diferencias entre el error numérico y las soluciones analíticas como
    función del ratio $\delta$ para varias funciones de densidad exponenciales.
    Estas comparaciones fueron llevadas a cabo para cada combinación entre los
    modelos de cascarón (Tabla~\ref{tab:shell-models}) y las grillas de cómputo
    (Tabla~\ref{tab:grids}) haciendo uso de un valor fijo de $D$. Cada curva
    corresponde a la diferencia máxima obtenida para cada modelo de cascarón
    para un valor particular de $b$ (ec.~\ref{eq:density-exp}). Las diferencias
    están reportadas como un porcentaje de las soluciones analíticas. Las
    líneas horizontales de a trazos y color negro representan el umbral de
    precisión de 0.1\%.
    }
\label{fig:delta-exponential}
\end{figure}


\subsection{Densidad sinusoidal}

Hasta ahora hemos analizado el comportamiento de la discretización basada en
densidad contra funciones de densidades lineales y exponenciales.
Sin embargo, este nuevo algoritmo es adecuado para ser utilizado con funciones
continuas más complejas, por ejemplo funciones no monotónicas o que presenten
múltiples puntos de inflexión.
Aunque este tipo de funciones de densidad son cuanto mucho raras de encontrar
en seteos geológicos, queremos someter nuestro algoritmo a estos casos con el
objetivo de mostrar de que puede resolver situaciones más complejas que
funciones lineales y exponenciales.
Además, obtener un valor óptimo del ratio $\delta$ para casos de funciones de
densidades irreales nos permitiría extrapolarlo a casos más simples
y realistas.
Es por esta razón que los análisis que se describirán a continuación están
destinados a someter el algoritmo a sus extremos y no para emular un escenario
real.

Consideremos modelos de cascarones específicos con una función de densidad
sinusoidal que se define de la siguiente manera:

\begin{equation}
    \rho(r') = A \sin \left( 2 \pi b \frac{r' - R}{R_2 - R_1} \right) + A,
    \label{eq:density-sine}
\end{equation}

\noindent donde $A$ es una constante que controla la amplitud y la traslación
vertical de la función seno, $R$ es el radio medio de la Tierra y $b$ es una
constante adimensional que regula cuántos períodos de la función trigonométrica
serán incluidos dentro de los radios interior y exterior.
La solución analítica para el potencial $V$ y su gradiente vertical $g_z$ que
genera un cascarón esférico con una densidad sinusoidal puede hallarse en el
Apéndice~\ref{cha:shell}.

Calculamos la diferencia relativa entre los resultados numéricos y la solución
analítica para $V$ y $g_z$ para cada combinación de modelos de cascarones
(Tabla~\ref{tab:shell-models}) y grillas de cómputo (Tabla~\ref{tab:grids}).
Hemos fijado el ratio distancia-tamaño $D$ a los valores obtenidos para el caso
de densidad lineal y exploramos valores de $\delta$ entre $10^{-4}$ y $1$. Los
cálculos fueron repetidos para valores de $b$ iguales a 1, 2, 5 y 10.
(Fig.~\ref{fig:sine-densities}). En todos los casos el valor de $A$ fue fijado
a 1650 kg/m$^3$.

\begin{figure}
\centering
\includegraphics[width=0.5\linewidth]{figs/tesseroids-variable-density/sine-densities.pdf}
\caption{
    Densidades sinusoidales asignadas a los cascarones esféricos durante la
    determinación del ratio $\delta$. Cada función de densidad corresponde a un
    valor distinto de $b$ en la ecuación~\ref{eq:density-sine}.
}
\label{fig:sine-densities}
\end{figure}

\begin{figure*}
\centering
\includegraphics[width=\linewidth]{figs/tesseroids-variable-density/sine-density-diffs.pdf}
\caption{
    Difference between the numerical and analytical solutions as a function of $\delta$
    ratio for different sinusoidal density functions.
    These comparisons were carried out for every shell model
    (Table~\ref{tab:shell-models}) and computation grid (Table~\ref{tab:grids})
    using a fixed value of $D$.
    Each curve corresponds to the maximum difference out of all shell models
    for a particular value of $b$ (Eq.~\ref{eq:density-sine}).
    Differences are reported as a percentage of the analytical solutions.
    The horizontal dashed black line represents a target difference of 0.1\%.
    }
\label{fig:delta-sine}
\end{figure*}

Fig.~\ref{fig:delta-sine} shows the relative differences between the analytical
and the numerical solutions for sinusoidal density case.
Once again, each curve corresponds to the maximum difference between all shell models.
For all values of $b$ except for $b=10$,
the differences fall below the 0.1\% threshold for $\delta = 0.1$.
In the case of $b=10$, a lower value of $\delta = 0.01$ is required to achieve 0.1\%
difference.
We note that, even for the case of $b=10$, the differences for $\delta = 0.1$ are below
1\%.


%%%%%%%%%%%%%%%%%%%%%%%%%%%%%%%%%%%%%%%%%%%%%%%%%%%%%%%%%%%%%%%%%%%%%%%%%%%%%%%

\section{Algorithm performance}

Because the density-based discretization algorithm introduces more divisions along the
radial dimension, it is reasonable that the computation time for variable densities will
be higher than the homogeneous density case.
Directly comparing the computation time of both cases cannot be done in a meaningful
way because it is highly dependent on the implementation of the algorithm,
the choice of programming language, and the particular density function used.
In order to obtain an indicator of the increase in the computation time,
we chose the number of tesseroid divisions in the density-based discretization as a
proxy measure.

We analyse exponential density functions (Eq.~\ref{eq:density-exp}) with values of $b$
equal to 1, 2, 5, 10, 30 and 100, and sinusoidal density functions
(Eq.~\ref{eq:density-sine}) with values of $b$ equal to 1, 2, 5, and 10.
We then apply the density-based discretization algorithm on each function and record the
number divisions carried out.
Figs.~\ref{fig:number-of-tesseroids}a and~\ref{fig:number-of-tesseroids}b show the
exponential and sinusoidal density functions and the resulting discretization points as
orange dots.

For the exponential density functions, the algorithm performs a single division
independently of the value of $b$.
Therefore, the computation time would likely be at least twice that of the homogeneous
density case (assuming equal implementations).
On the other hand, Fig.~\ref{fig:number-of-tesseroids}c shows an almost linear relation
between the number of divisions in the sinusoidal density case and the value of $b$.
Thus, the computation time is likely to be dependent on the number of wavelengths
of the sinusoidal function that are contained within the tesseroid.

\begin{figure*}
\centering
\includegraphics[width=\linewidth]{figs/tesseroids-variable-density/number-of-tesseroids.pdf}
\caption{
    Number of divisions performed by the density-based discretization algorithm
    (with $\delta = 0.1$) in case of
    (a)~exponential density functions with the same values of $b$ shown in
    Fig.~\ref{fig:exp-densities},
    (b)~sinusoidal density functions with the same values of $b$ shown in
    Fig.~\ref{fig:sine-densities}.
    On both figures, the locations of the tesseroid divisions are are marked with orange
    dots.
    (c) shows the number of discretized tesseroids for the sinusoidal density functions
    as a function of $b$.
}
\label{fig:number-of-tesseroids}
\end{figure*}


%%%%%%%%%%%%%%%%%%%%%%%%%%%%%%%%%%%%%%%%%%%%%%%%%%%%%%%%%%%%%%%%%%%%%%%%%%%%%%%

\section{Application to the Neuquén Basin}

We applied the new algorithms and optimal values of $D$ and $\delta$ determined
previously to calculate the gravitational effects of the Neuquén Basin,
a sedimentary basin located to the east of the Andes, between 32$^\circ$S and
40$^\circ$S latitude (Fig.~\ref{fig:neuquen-basin}a).
The basin includes continental and marine siliciclastics, carbonates, and evaporites
accumulated over the Jurassic and the Cretaceous constituting a stratigraphic record up
to 5000m of depth \citep{howell2005}.

\begin{figure*}
\centering
\includegraphics[width=\linewidth]{figs/tesseroids-variable-density/neuquen-basin.pdf}
\caption{
    Gravitational effects of the Neuquén sedimentary basin modelled
    using tesseroids with an exponential density function of depth.
    (a)~Topography of the Neuquén Basin (in km) and its location in South America,
    (b)~thickness of the sedimentary basin \citep[in meters;][]{heine2007},
    (c)~resulting gravitational potential $V$,
    (d)~resulting vertical component of the gradient ($g_z$),
    calculated at 10km of height over the mean Earth radius.
}
\label{fig:neuquen-basin}
\end{figure*}

\begin{figure}
\centering
%\iftwocol{
\includegraphics[width=\linewidth]{figs/tesseroids-variable-density/neuquen-basin-densities.pdf}
%}{
%\includegraphics[width=0.5\linewidth]{figs/tesseroids-variable-density/neuquen-basin-densities.pdf}
%}
\caption{
    Linear and exponential densities used to compute the gravitational fields generated
    by a tesseroid model of the Neuquén sedimentary basin.
    The height is defined above the mean Earth radius, and its axis is spanned between
    the deepest and the highest point of the basin.
}
\label{fig:neuquen-basin-densities}
\end{figure}


\begin{figure*}
\centering
\includegraphics[width=\linewidth]{figs/tesseroids-variable-density/neuquen-basin-diffs.pdf}
\caption{
    Differences between the gravitational fields generated by the tesseroid model of the
    Neuquén basin with an exponential density contrast and with a homogeneous and
    linear density variation.
    \mbox{(a)-(b)}~Differences on $V$ and $g_z$ between the exponential density model
    and the homogeneous density one,
    \mbox{(c)-(f)}~differences on $V$ and $g_z$ between the exponential density model
    and the linear density one, calculated at 10km of height over the mean Earth radius.
}
\label{fig:neuquen-basin-diffs}
\end{figure*}

The thickness of the sediment pack was digitized from \citet{heine2007} on a regular
grid with a resolution of 0.05$^\circ$ on both longitude and latitude directions
(Fig.~\ref{fig:neuquen-basin}b).
We created a tesseroid model of the sediment pack by placing a
$0.05^\circ \times 0.05^\circ$ tesseroid on each node of the grid.
The top of each tesseroid was fixed at the median of the topography of the basin
(845~m above mean Earth radius) and the bottom at corresponding thickness of the basin.

We must also define a density function for the tesseroid model.
\citet{sigismondi2012} measured a minimum and maximum density contrast for
the Neuquén basin of -412kg/m$^3$ and -275kg/m$^3$, respectively.
We have chosen an exponential density variation (Eq.~\ref{eq:density-exp}) that assumes
the minimum value on the top surface and the maximum at 5014m depth (the thickest part
of the basin), with a value of $b$ equal to 3.
This density variation is in the order of magnitude of the ones used by
\citet{cowie1990} and \citet{cordell1973}.
This density function can be seen on Fig.~\ref{fig:neuquen-basin-densities}.

Finally, we computed the gravitational potential $V$ and the vertical component of the
gradient ($g_z$) on a computation grid of $159\times163$ nodes ($0.05^\circ$ spacing on
both longitude and latitude) at a 10 km height over the mean Earth radius.
The resulting fields can be seen in Fig.~\ref{fig:neuquen-basin}c-d.

We computed the differences between the results for the exponential density function and
those generated by the same model but now with a constant density (the mean of
-412kg/m$^3$ and -275kg/m$^3$).
We also computed the differences between the exponential density and a linear density
that assumes these two values on the highest and the deepest point of the basin
(Fig.~\ref{fig:neuquen-basin-densities}).
Fig.~\ref{fig:neuquen-basin-diffs}a-b and Fig.~\ref{fig:neuquen-basin-diffs}c-d show the
differences with the constant and the linear density, respectively.
The maximum absolute difference in the computed $g_z$ is approximately 8 mGal for the
homogeneous case and approximately 6 mGal for the linear density case.
Both are well above the accuracy of most available data products.


%%%%%%%%%%%%%%%%%%%%%%%%%%%%%%%%%%%%%%%%%%%%%%%%%%%%%%%%%%%%%%%%%%%%%%%%%%%%%%%

\section{Discussion}

By including the density function into the GLQ integration, the method described here
can be applied without modification to tesseroids with density following any continuous
function of radius.
A density-based discretization algorithm divides the tesseroid in the radial dimension
to ensure accurate integration of the density function.
This algorithm is independent of the GLQ integration and could potentially be used to
determine an optimal discretization when approximating a density function by piecewise
linear \citep{lin2019} or piecewise polynomial \citep{fukushima2018} functions.

Our numerical experiments show that the two dimensional adaptive discretization is
enough to achieve 0.1\% accuracy with a second-order GLQ in the case of a linear density
function (Fig.~\ref{fig:D-linear}).
Values of the distance-size ratio determined here for the gravitational potential
($D=1$) and its vertical derivative ($D=2.5$) are compatible with \citet{uieda2016}.
These results also show that there is no significant relationship between the accuracy
of the method and the thickness of the tesseroid model.

The $D$-$\delta$ space exploration for the exponential density function
(Fig.~\ref{fig:grid-search}) showed that values of $D$ determined for the linear density
case are equally applicable to the exponential density case.
Furthermore, the difference with respect to the analytical solution only falls below
0.1\% for values of $\delta$ lower than 0.1.
It follows that the density-based discretization is required to achieve the desired
accuracy level for non-linear density functions.

More detailed error analyses showed that $\delta = 0.1$ is sufficient to guarantee an
accuracy of 0.1\% for all exponential density functions tested
(Fig.~\ref{fig:delta-exponential})
and most of the sinusoidal functions tested (Fig.~\ref{fig:delta-sine}).
The exception is the case of a sinusoidal function with $b = 10$
(Eq.~\ref{eq:density-sine}), for which $\delta = 0.01$ is required to achieve 0.1\%
accuracy.
Nevertheless, using $\delta = 0.1$ in this case would still produce results with an
error of less than 1\%.

While the algorithms proposed here can be applied to any continuous density function,
the optimal values for $D$ and $\delta$ are empirically determined only for linear,
exponential, and sinusoidal density functions.
Thus, we can only say with certainty that these values will produce results with 0.1\%
accuracy for these functions.
However, all of our numerical tests include worse-case scenarios (zero computation
heights, large tesseroids, highly variable density functions, etc).
For this reason, these optimal values of $D$ and $\delta$ can plausibly be
extrapolated to any realistic continuous density function.
Notwithstanding, we encourage users of the algorithms and software to perform similar
tests in order evaluate the accuracy when using density functions more complex than the
ones tested here.

The algorithm performance analysis shows that the computation time using variable
densities is likely to be at least two times larger than the homogeneous density case.
It is also likely to increase proportionally to the number of inflection points of the
density function.
As is the case for most numerical methods, there is a trade-off between computation time
and accuracy.
Nevertheless, density profiles will have few inflection points for most geophysical
applications.
Therefore, the computation time would be in the same order of
magnitude as the one for the homogeneous density in most real world applications.

%%%%%%%%%%%%%%%%%%%%%%%%%%%%%%%%%%%%%%%%%%%%%%%%%%%%%%%%%%%%%%%%%%%%%%%%%%%%%%%

\section{Conclusions}

We have developed a new methodology to compute the gravitational potential and its
gradient for a tesseroid with a density given by a continuous function of radius.
It numerically solves the volume integrals through Gauss-Legendre Quadrature (GLQ) by
including the density function in the numerical integration.
By implementing the algorithm in the dynamic Python programming language, users can
define their own density function to supply to the software.
This allows the use of any arbitrary continuous function without modification to the
method or software.
The accuracy of the numerical integration is automatically controlled by a two
dimensional adaptive discretization algorithm and a new density-based discretization
algorithm.
The former divides the tesseroid in half if the ratios of the distance to the
computation point and the latitudinal or the longitudinal size of the tesseroid are
smaller than a predefined distance-size ratio $D$.
This algorithm minimises the integration error when the computation point is close to
the tesseroid.
Nevertheless, the adaptive discretization alone is not sufficient to guarantee the
accuracy of the method in case of tesseroids with non-linear density functions.
To overcome this challenge, the density-based discretization algorithm
divides the tesseroid along the radial dimension on the points at which the maximum
variations of the density function take place.
The number of radial divisions performed, and thus the accuracy of the computation, is
controlled by the $\delta$ parameter.
This new algorithm is intended to minimise the error due to the inability of
the GLQ to produce precise approximations of density functions with sharp variations.

We empirically determined optimal values for $D$ and $\delta$ by comparing the numerical
results with analytical solutions.
Our analysis included a range of tesseroid models and computation grids as well as
common density functions.
These values minimize the computational load while maintaining the numerical error below
0.1\% of an analytical solution.
The density functions used to establish the optimal values were a linear, an
exponential, and a sinusoidal function.
The linear function represents the smoothest variation of density and does not require
density-based discretization at all.
Through this analysis, we have obtained optimal values for the distance-size ratio $D$
of 1 and 2.5 for the potential and its gradient, respectively.
We analysed the error for exponential functions ranging from smooth to sharp and
sinusoidal function of different wavelengths to test the accuracy of the density-based
discretization.
It worth noting that the sinusoidal density tests were carried out in order to test the
algorithm to the extreme and not to emulate a real world scenario.
Values of $\delta$ equal to 0.1 are sufficient to guarantee a 0.1\% accuracy for most
realist cases tested.
These results could be extrapolated to other continuous density functions that are
sufficiently smooth, as is the case for most common geophysical applications.
Smaller values of $\delta$ can be used for more highly variable density functions to
increase the accuracy of the integration accordingly.

Some computational performance is sacrificed in order to obtain an accurate and general
purpose method.
The density-based discretization adds to the computation time by increasing
the number of GLQ integrations.
Likewise, allowing users to supply custom density functions incurs an overhead cost
because of the added function evaluations.
It is also impossible to optimize the source code and formulas without knowledge of the
specific density function.
Nonetheless, the density-based discretization is independent of the adaptive
discretization and GLQ integration.
It can be viewed as a type of pre-processing step that can be combined with other more
specialized integration methods.

An application to modelling the Neuquén Basin, Argentina, demonstrates that the
effects of sediment compaction cannot be ignored.
When compared to an exponential density function, a homogeneous density with depth would
result in an error of up to -8 mGal, while a linear density function would result in up
to -6 mGal error.
Accurate and robust forward modelling is a key component to any gravity inversion
method.
Errors of this magnitude could result in significant overestimation of sedimentary
basin thickness, for example.



%%%%%%%%%%%%%%%%%%%%%%%%%%%%%%%%%%%%%%%%%%%%%%%%%%%%%%%%%%%%%%%%%%%%%%%%%%%%%%%

\section{Acknowledgments}

We are indebted to the developers and maintainers of the open-source software without
which this work would not have been possible.
We would also like to thank editor Kosuke Heki and two anonymous reviewers for their
comments and contributions.
This is SOEST contribution 10717.
