\chapter{Soluciones analíticas para un cascarón esférico}
\label{cha:shell}

Consideremos un cascarón esférico con radio interior $R_1$ y radio exterior
$R_2$, cuya densidad es función $\rho(r')$ de la coordenada radial.
Deseamos obtener una expresión analítica para los campos gravitatorios que
general en cualquier punto externo ubicado a una distancia $r$ del centro del
cascarón ($r > R_2$).

Según el Teorema del Cascarón de Newton (\emph{Newton's Shell Theorem, Theorem
XXXI}) \citep{chandrasekhar1995, binney2008}, el potencial gravitatorio
generado por el cascarón esférico en cualquier punto externo es equivalente
al que generaría si toda su masa estuviera concentrada en un punto localizado
en su centro:

\begin{equation}
    V_\text{sh}(\phi, \lambda, r) = \frac{GM}{r},
\end{equation}

\noindent donde $M$ es la masa total del cascarón, la cual puede ser fácilmente
calculada como:

\begin{equation}
    M =
    \iiint\limits_{\Omega} \rho(r') dV =
    4\pi \int\limits_{R_1}^{R_2} \rho(r') {r'}^2 dr',
\end{equation}

\noindent donde $\Omega$ simboliza el volumen del cascarón.

Combinando las dos ecuaciones anteriores, obtenemos la siguiente expresión para
el potencial:

\begin{equation}
    V_\text{sh}(r) = \frac{4\pi G}{r}
    \int\limits_{R_1}^{R_2} {r'}^2 \rho(r') \, dr',
\label{eq:shell-pot}
\end{equation}

\noindent la cual es equivalente a la obtenida por \citet[p.62]{binney2008}.

El gradiente de potenciales que dependen solo de $r$ poseen solo una componente
no nula: la componente vertical del gradiente ($g_z$).
Según \citet{grombein2013}:

\begin{equation}
    g_z(r) = \frac{V_\text{sh}(r)}{r}.
\label{eq:shell-gz}
\end{equation}

A partir de la ecuación~\ref{eq:shell-pot} podemos obtener expresiones del
potencial gravitatorio para diferentes funciones de densidad. La integración de
las siguientes funciones de densidad fueron llevadas a cabo mediante el uso de
SymPy \citep{sympy2017}, una librería de Python para matemática simbólica.

\section{Densidad lineal}

Para una densidad lineal

\begin{equation}
    \rho(r') = ar' + b\ ,
\end{equation}

\noindent
el potencial gravitatorio en cualquier punto externo es

\begin{equation}
    V_\text{sh}^\text{lin}(r) = \pi G \left[
    a \frac{R_2^4 - R_1^4}{r} +
    b \,\frac{4}{3} \frac{R_2^3 - R_1^3}{r} \right].
    \label{eq:shell-pot-linear}
\end{equation}

\noindent El primer término de la ecuación reproduce el potencial generado por
un cascarón esférico con densidad variable $\rho(r') = ar'$, mientras que el
segundo término es equivalente al potencial generado por el mismo cascarón con
densidad homogénea $\rho = b$ \citep{mikuska2006,grombein2013}.
La ecuación~\ref{eq:shell-pot-linear} coincide con la obtenida por \citet{lin2019}.

\section{Densidad exponencial}

Para una densidad exponencial

\begin{equation}
    \rho(r') = A e^{- k (r' - R)},
\end{equation}

\noindent donde $A$, $k$ y $R$ son constantes, el potencial gravitatorio en
cualquier punto exterior es

\begin{equation}
    \begin{split}
        V_\text{exp}(r) = \frac{4\pi G}{r} \frac{A}{k^3} \Big[
        & \left( R_1^2 k^2 + 2 R_1 k + 2 \right) e^{- k (R_1 - R)} - \\
        & \left( R_2^2 k^2 + 2 R_2 k + 2 \right) e^{- k (R_2 - R)}
        \Big].
    \end{split}
\end{equation}


\section{Densidad sinusoidal}

Para una función de densidad sinusoidal

\begin{equation}
    \rho(r') = A \sin ( k (r' - R)),
\end{equation}

\noindent donde $A$, $k$ y $R$ son constantes, el potencial gravitatorio en
cualquier punto exterior es

\begin{equation}
    \begin{split}
        V_\text{sine}(r) = \frac{4\pi G}{r} \frac{A}{k^3} \Big[
    & (2 - k^2 R_2^2) \cos(k(R_2 - R)) + 2 k R_2 \sin(k(R_2 - R)) - \\
    & (2 - k^2 R_1^2) \cos(k(R_1 - R)) - 2 k R_1 \sin(k(R_1 - R))
    \Big].
    \end{split}
\end{equation}
