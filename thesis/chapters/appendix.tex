\chapter{Soluciones analíticas para un cascarón esférico}
\label{cha:shell}

Consideremos un cascarón esférico con radio interior $R_1$ y radio exterior
$R_2$, cuya densidad es función $\rho(r')$ de la coordenada radial.
Deseamos obtener una expresión analítica para los campos gravitatorios que
general en cualquier punto externo ubicado a una distancia $r$ del centro del
cascarón ($r > R_2$).

Según el Teorema del Cascarón de Newton (\emph{Newton's Shell Theorem, Theorem
XXXI}) \citep{chandrasekhar1995, binney2008}, el potencial gravitatorio
generado por el cascarón esférico en cualquier punto externo es equivalente
al que generaría si toda su masa estuviera concentrada en un punto localizado
en su centro:

\begin{equation}
    V_\text{sh}(\phi, \lambda, r) = \frac{GM}{r},
\end{equation}

\noindent donde $M$ es la masa total del cascarón, la cual puede ser fácilmente
calculada como:

\begin{equation}
    M =
    \iiint\limits_{\Omega} \rho(r') dV =
    4\pi \int\limits_{R_1}^{R_2} \rho(r') {r'}^2 dr',
\end{equation}

\noindent donde $\Omega$ simboliza el volumen del cascarón.

Combinando las dos ecuaciones anteriores, obtenemos la siguiente expresión para
el potencial:

\begin{equation}
    V_\text{sh}(r) = \frac{4\pi G}{r}
    \int\limits_{R_1}^{R_2} {r'}^2 \rho(r') \, dr',
\label{eq:shell-pot}
\end{equation}

\noindent la cual es equivalente a la obtenida por \citet[p.62]{binney2008}.

El gradiente de potenciales que dependen solo de $r$ poseen solo una componente
no nula: la componente vertical del gradiente ($g_z$).
Según \citet{grombein2013}:

\begin{equation}
    g_z(r) = \frac{V_\text{sh}(r)}{r}.
\label{eq:shell-gz}
\end{equation}

A partir de la ecuación~\ref{eq:shell-pot} podemos obtener expresiones del
potencial gravitatorio para diferentes funciones de densidad. La integración de
las siguientes funciones de densidad fueron llevadas a cabo mediante el uso de
SymPy \citep{sympy2017}, una librería de Python para matemática simbólica.

\section{Densidad lineal}

Para una densidad lineal

\begin{equation}
    \rho(r') = ar' + b\ ,
\end{equation}

\noindent
el potencial gravitatorio en cualquier punto externo es

\begin{equation}
    V_\text{sh}^\text{lin}(r) = \pi G \left[
    a \frac{R_2^4 - R_1^4}{r} +
    b \,\frac{4}{3} \frac{R_2^3 - R_1^3}{r} \right].
    \label{eq:shell-pot-linear}
\end{equation}

\noindent El primer término de la ecuación reproduce el potencial generado por
un cascarón esférico con densidad variable $\rho(r') = ar'$, mientras que el
segundo término es equivalente al potencial generado por el mismo cascarón con
densidad homogénea $\rho = b$ \citep{mikuska2006,grombein2013}.
La ecuación~\ref{eq:shell-pot-linear} coincide con la obtenida por \citet{lin2019}.

\section{Densidad exponencial}

Para una densidad exponencial

\begin{equation}
    \rho(r') = A e^{- k (r' - R)},
\end{equation}

\noindent donde $A$, $k$ y $R$ son constantes, el potencial gravitatorio en
cualquier punto exterior es

\begin{equation}
    \begin{split}
        V_\text{exp}(r) = \frac{4\pi G}{r} \frac{A}{k^3} \Big[
        & \left( R_1^2 k^2 + 2 R_1 k + 2 \right) e^{- k (R_1 - R)} - \\
        & \left( R_2^2 k^2 + 2 R_2 k + 2 \right) e^{- k (R_2 - R)}
        \Big].
    \end{split}
\end{equation}


\section{Densidad sinusoidal}

Para una función de densidad sinusoidal

\begin{equation}
    \rho(r') = A \sin ( k (r' - R)),
\end{equation}

\noindent donde $A$, $k$ y $R$ son constantes, el potencial gravitatorio en
cualquier punto exterior es

\begin{equation}
    \begin{split}
        V_\text{sine}(r) = \frac{4\pi G}{r} \frac{A}{k^3} \Big[
    & (2 - k^2 R_2^2) \cos(k(R_2 - R)) + 2 k R_2 \sin(k(R_2 - R)) - \\
    & (2 - k^2 R_1^2) \cos(k(R_1 - R)) - 2 k R_1 \sin(k(R_1 - R))
    \Big].
    \end{split}
\end{equation}


\chapter[%
    Parámetros para localizar fuentes equivalentes en pruebas con datos sintéticos
]{%
    Parámetros para localizar fuentes equivalentes en pruebas con datos sintéticos
}
\chaptermark{Parámetros de fuentes equivalentes}
\label{cha:sources}

Las tablas~\ref{tab:parameters-ground-survey}
y~\ref{tab:parameters-airborne-survey} muestran valores de parámetros
utilizados para analizar estrategias de distribución de fuentes equivalentes
con datos sintéticos (sección~\ref{sec:synthetic_distributions}, junto con los
valores óptimos correspondientes a cada estrategia.
Estos valores óptimos fueron utilizados para producir los resultados de las
figuras~\ref{fig:ground-survey-differences}
y~\ref{fig:airborne-survey-differences}.

{\renewcommand\normalsize{\scriptsize}%
\normalsize

\begin{table}
    \centering
    \caption{
        Parámetros utilizados para producir cada distribución de fuentes
        durante la interpolación de datos sintéticos sobre terreno.
        Contiene además el conjunto de parámetros que genera el menor
        \acs{RMSE} para cada estrategia de distribución de fuentes y su
        correspondiente valor.
    }
    \label{tab:parameters-ground-survey}
    \begin{tabular}{c c l c c c}
        \textbf{Distribución}
            & \textbf{Profundidad}
            & \multicolumn{1}{c}{\textbf{Parámetros}}
            & \textbf{Valores}
            & \textbf{Óptimo}
            & \textbf{RMSE} \\
        \toprule

        \multirow{8}{*}{Debajo de datos}
            & \multirow{2}{*}{Constante}
                & Profundidad (m)
                & \GroundSourceBelowDataConstantDepthDepth
                & \BestGroundSourceBelowDataConstantDepthDepth
                & \multirow{2}{*}{
                    \BestGroundSourceBelowDataConstantDepthRms
                  } \\
            &
                & Amortiguamiento
                & \GroundSourceBelowDataConstantDepthDamping
                & \BestGroundSourceBelowDataConstantDepthDamping
                & \\
            \cmidrule{2-6}
            & \multirow{2}{*}{Relativa}
                & Profundidad (m)
                & \GroundSourceBelowDataRelativeDepthDepth
                & \BestGroundSourceBelowDataRelativeDepthDepth
                & \multirow{2}{*}{
                    \BestGroundSourceBelowDataRelativeDepthRms
                  } \\
            &
                & Amortiguamiento
                & \GroundSourceBelowDataRelativeDepthDamping
                & \BestGroundSourceBelowDataRelativeDepthDamping
                & \\
            \cmidrule{2-6}
            & \multirow{4}{*}{Variable}
                & Profundidad (m)
                & \GroundSourceBelowDataVariableDepthDepth
                & \BestGroundSourceBelowDataVariableDepthDepth
                & \multirow{4}{*}{
                    \BestGroundSourceBelowDataVariableDepthRms
                  } \\
            &
                & Factor de profundidad
                & \GroundSourceBelowDataVariableDepthDepthFactor
                & \BestGroundSourceBelowDataVariableDepthDepthFactor
                & \\
            &
                & $k$ vecinos cercanos
                & \GroundSourceBelowDataVariableDepthKNearest
                & \BestGroundSourceBelowDataVariableDepthKNearest
                & \\
            &
                & Amortiguamiento
                & \GroundSourceBelowDataVariableDepthDamping
                & \BestGroundSourceBelowDataVariableDepthDamping
                & \\
        \midrule

        \multirow{11}{*}{Promed. por bloque}
            & \multirow{3}{*}{Constante}
                & Profundidad (m)
                & \GroundBlockAveragedSourcesConstantDepthDepth
                & \BestGroundBlockAveragedSourcesConstantDepthDepth
                & \multirow{3}{*}{
                    \BestGroundBlockAveragedSourcesConstantDepthRms
                  } \\
            &
                & Tamaño de bloque (m)
                & \GroundBlockAveragedSourcesConstantDepthSpacing
                & \BestGroundBlockAveragedSourcesConstantDepthSpacing
                & \\
            &
                & Amortiguamiento
                & \GroundBlockAveragedSourcesConstantDepthDamping
                & \BestGroundBlockAveragedSourcesConstantDepthDamping
                & \\
            \cmidrule{2-6}
            & \multirow{3}{*}{Relativa}
                & Profundidad (m)
                & \GroundBlockAveragedSourcesRelativeDepthDepth
                & \BestGroundBlockAveragedSourcesRelativeDepthDepth
                & \multirow{3}{*}{
                    \BestGroundBlockAveragedSourcesRelativeDepthRms
                  } \\
            &
                & Tamaño de bloque (m)
                & \GroundBlockAveragedSourcesRelativeDepthSpacing
                & \BestGroundBlockAveragedSourcesRelativeDepthSpacing
                & \\
            &
                & Amortiguamiento
                & \GroundBlockAveragedSourcesRelativeDepthDamping
                & \BestGroundBlockAveragedSourcesRelativeDepthDamping
                & \\
            \cmidrule{2-6}
            & \multirow{5}{*}{Variable}
                & Profundidad (m)
                & \GroundBlockAveragedSourcesVariableDepthDepth
                & \BestGroundBlockAveragedSourcesVariableDepthDepth
                & \multirow{5}{*}{
                    \BestGroundBlockAveragedSourcesVariableDepthRms
                  } \\
            &
                & Factor de profundidad
                & \GroundBlockAveragedSourcesVariableDepthDepthFactor
                & \BestGroundBlockAveragedSourcesVariableDepthDepthFactor
                & \\
            &
                & $k$ vecinos cercanos
                & \GroundBlockAveragedSourcesVariableDepthKNearest
                & \BestGroundBlockAveragedSourcesVariableDepthKNearest
                & \\
            &
                & Tamaño de bloque (m)
                & \GroundBlockAveragedSourcesVariableDepthSpacing
                & \BestGroundBlockAveragedSourcesVariableDepthSpacing
                & \\
            &
                & Amortiguamiento
                & \GroundBlockAveragedSourcesVariableDepthDamping
                & \BestGroundBlockAveragedSourcesVariableDepthDamping
                & \\
        \midrule

        \multirow{4}{*}{Grilla regular}
            & \multirow{4}{*}{Constante}
                & Profundidad (m)
                & \GroundGridSourcesConstantDepthDepth
                & \BestGroundGridSourcesConstantDepthDepth
                & \multirow{4}{*}{
                    \BestGroundGridSourcesConstantDepthRms
                  } \\
            &
                & Espaciado de grilla (m)
                & \GroundGridSourcesConstantDepthSpacing
                & \BestGroundGridSourcesConstantDepthSpacing
                & \\
            &
                & Amortiguamiento
                & \GroundGridSourcesConstantDepthDamping
                & \BestGroundGridSourcesConstantDepthDamping
                & \\
    \end{tabular}
\end{table}

\begin{table}
    \centering
    \caption{
        Parámetros utilizados para producir cada distribución de fuentes
        durante la interpolación de datos sintéticos aéreos.
        Contiene además el conjunto de parámetros que genera el menor
        \acs{RMSE} para cada estrategia de distribución de fuentes y su
        correspondiente valor.
    }
    \label{tab:parameters-airborne-survey}
    \begin{tabular}{c c l c c c}
        \textbf{Distribución}
            & \textbf{Profundidad}
            & \multicolumn{1}{c}{\textbf{Parámetros}}
            & \textbf{Valores}
            & \textbf{Óptimo}
            & \textbf{RMSE} \\
        \toprule

        \multirow{8}{*}{Debajo de datos}
            & \multirow{2}{*}{Constante}
                & Profundidad (m)
                & \AirborneSourceBelowDataConstantDepthDepth
                & \BestAirborneSourceBelowDataConstantDepthDepth
                & \multirow{2}{*}{
                    \BestAirborneSourceBelowDataConstantDepthRms
                  } \\
            &
                & Amortiguamiento
                & \AirborneSourceBelowDataConstantDepthDamping
                & \BestAirborneSourceBelowDataConstantDepthDamping
                & \\
            \cmidrule{2-6}
            & \multirow{2}{*}{Relativa}
                & Profundidad (m)
                & \AirborneSourceBelowDataRelativeDepthDepth
                & \BestAirborneSourceBelowDataRelativeDepthDepth
                & \multirow{2}{*}{
                    \BestAirborneSourceBelowDataRelativeDepthRms
                  } \\
            &
                & Amortiguamiento
                & \AirborneSourceBelowDataRelativeDepthDamping
                & \BestAirborneSourceBelowDataRelativeDepthDamping
                & \\
            \cmidrule{2-6}
            & \multirow{4}{*}{Variable}
                & Profundidad (m)
                & \AirborneSourceBelowDataVariableDepthDepth
                & \BestAirborneSourceBelowDataVariableDepthDepth
                & \multirow{4}{*}{
                    \BestAirborneSourceBelowDataVariableDepthRms
                  } \\
            &
                & Factor de profundidad
                & \AirborneSourceBelowDataVariableDepthDepthFactor
                & \BestAirborneSourceBelowDataVariableDepthDepthFactor
                & \\
            &
                & $k$ vecinos cercanos
                & \AirborneSourceBelowDataVariableDepthKNearest
                & \BestAirborneSourceBelowDataVariableDepthKNearest
                & \\
            &
                & Amortiguamiento
                & \AirborneSourceBelowDataVariableDepthDamping
                & \BestAirborneSourceBelowDataVariableDepthDamping
                & \\
        \midrule

        \multirow{11}{*}{Prom. por bloque}
            & \multirow{3}{*}{Constante}
                & Profundidad (m)
                & \AirborneBlockAveragedSourcesConstantDepthDepth
                & \BestAirborneBlockAveragedSourcesConstantDepthDepth
                & \multirow{3}{*}{
                    \BestAirborneBlockAveragedSourcesConstantDepthRms
                  } \\
            &
                & Tamaño de bloque (m)
                & \AirborneBlockAveragedSourcesConstantDepthSpacing
                & \BestAirborneBlockAveragedSourcesConstantDepthSpacing
                & \\
            &
                & Amortiguamiento
                & \AirborneBlockAveragedSourcesConstantDepthDamping
                & \BestAirborneBlockAveragedSourcesConstantDepthDamping
                & \\
            \cmidrule{2-6}
            & \multirow{3}{*}{Relativa}
                & Profundidad (m)
                & \AirborneBlockAveragedSourcesRelativeDepthDepth
                & \BestAirborneBlockAveragedSourcesRelativeDepthDepth
                & \multirow{3}{*}{
                    \BestAirborneBlockAveragedSourcesRelativeDepthRms
                  } \\
            &
                & Tamaño de bloque (m)
                & \AirborneBlockAveragedSourcesRelativeDepthSpacing
                & \BestAirborneBlockAveragedSourcesRelativeDepthSpacing
                & \\
            &
                & Amortiguamiento
                & \AirborneBlockAveragedSourcesRelativeDepthDamping
                & \BestAirborneBlockAveragedSourcesRelativeDepthDamping
                & \\
            \cmidrule{2-6}
            & \multirow{5}{*}{Variable}
                & Profundidad (m)
                & \AirborneBlockAveragedSourcesVariableDepthDepth
                & \BestAirborneBlockAveragedSourcesVariableDepthDepth
                & \multirow{5}{*}{
                    \BestAirborneBlockAveragedSourcesVariableDepthRms
                  } \\
            &
                & Factor de profundidad
                & \AirborneBlockAveragedSourcesVariableDepthDepthFactor
                & \BestAirborneBlockAveragedSourcesVariableDepthDepthFactor
                & \\
            &
                & $k$ vecinos cercanos
                & \AirborneBlockAveragedSourcesVariableDepthKNearest
                & \BestAirborneBlockAveragedSourcesVariableDepthKNearest
                & \\
            &
                & Tamaño de bloque (m)
                & \AirborneBlockAveragedSourcesVariableDepthSpacing
                & \BestAirborneBlockAveragedSourcesVariableDepthSpacing
                & \\
            &
                & Amortiguamiento
                & \AirborneBlockAveragedSourcesVariableDepthDamping
                & \BestAirborneBlockAveragedSourcesVariableDepthDamping
                & \\
        \midrule

        \multirow{4}{*}{Grilla regular}
            & \multirow{4}{*}{Constante}
                & Profundidad (m)
                & \AirborneGridSourcesConstantDepthDepth
                & \BestAirborneGridSourcesConstantDepthDepth
                & \multirow{4}{*}{
                    \BestAirborneGridSourcesConstantDepthRms
                  } \\
            &
                & Espaciado de grilla (m)
                & \AirborneGridSourcesConstantDepthSpacing
                & \BestAirborneGridSourcesConstantDepthSpacing
                & \\
            &
                & Amortiguamiento
                & \AirborneGridSourcesConstantDepthDamping
                & \BestAirborneGridSourcesConstantDepthDamping
                & \\
    \end{tabular}
\end{table}
}
