\chapter{Analytical Solutions for Spherical Shell}
\label{cha:shell}

Consider a spherical shell with inner radius $R_1$ and outer radius $R_2$,
whose density is a function $\rho(r')$ of the radial coordinate.
We want to get an analytical expression for the gravitational fields the shell generates
on any external point located at a distance $r$ from the centre of the shell
($r > R_2$).

According to Newton's Shell Theorem (Theorem XXXI)
\citep{chandrasekhar1995, binney2008}, the gravitational potential generated by the
shell on any external point is the same as it would be if its mass were concentrated on
a point located at its centre:

\begin{equation}
    V_\text{sh}(\phi, \lambda, r) = \frac{GM}{r},
\end{equation}

\noindent where $M$ is the total mass of the shell, which can be easily computed as
follows:

\begin{equation}
    M =
    \iiint\limits_{\Omega} \rho(r') dV =
    4\pi \int\limits_{R_1}^{R_2} \rho(r') {r'}^2 dr',
\end{equation}

\noindent where $\Omega$ symbolizes the volume of the shell.

Combining the two previous equations, we get the following expression for the potential:

\begin{equation}
    V_\text{sh}(r) = \frac{4\pi G}{r}
    \int\limits_{R_1}^{R_2} {r'}^2 \rho(r') \, dr',
\label{eq:shell-pot}
\end{equation}

\noindent which is in agreement with the one obtained by \citet[p.62]{binney2008}.

The gradient of potentials that depends solely on $r$ have only one non zero components:
the vertical component of the gradient ($g_z$).
Following \citet{grombein2013}:

\begin{equation}
    g_z(r) = \frac{V_\text{sh}(r)}{r}.
\label{eq:shell-gz}
\end{equation}

From Eq.~\ref{eq:shell-pot} we can obtain expressions of the gravitational potential for
different density functions.
The integration of the following density functions have been carried out by SymPy
\citep{sympy2017}, a Python library for symbolic mathematics.

\section{Linear density}

For a linear density function

\begin{equation}
    \rho(r') = ar' + b\ ,
\end{equation}

\noindent
the gravitational potential at any external point is

\begin{equation}
    V_\text{sh}^\text{lin}(r) = \pi G \left[
    a \frac{R_2^4 - R_1^4}{r} +
    b \,\frac{4}{3} \frac{R_2^3 - R_1^3}{r} \right].
    \label{eq:shell-pot-linear}
\end{equation}

\noindent The first term on this equation reproduces the potential generated
by a spherical shell with variable density $\rho(r') = ar'$, while the second
term constitutes the potential generated by a spherical shell with homogeneous
density $\rho = b$ \citep{mikuska2006,grombein2013}.
Eq~\ref{eq:shell-pot-linear} is in agreement with the one obtained by \citet{lin2019}.

\section{Exponential density}

For an exponential density function

\begin{equation}
    \rho(r') = A e^{- k (r' - R)},
\end{equation}

\noindent
where $A$, $k$ and $R$ are constants, the gravitational potential on any external point
is

\begin{equation}
    \begin{split}
        V_\text{exp}(r) = \frac{4\pi G}{r} \frac{A}{k^3} \Big[
        & \left( R_1^2 k^2 + 2 R_1 k + 2 \right) e^{- k (R_1 - R)} - \\
        & \left( R_2^2 k^2 + 2 R_2 k + 2 \right) e^{- k (R_2 - R)}
        \Big].
    \end{split}
\end{equation}


\section{Sinusoidal density}

For an sinusoidal density function

\begin{equation}
    \rho(r') = A \sin ( k (r' - R)),
\end{equation}

\noindent
where $A$, $k$ and $R$ are constants, the gravitational potential on any external point
is

\begin{equation}
    \begin{split}
        V_\text{sine}(r) = \frac{4\pi G}{r} \frac{A}{k^3} \Big[
    & (2 - k^2 R_2^2) \cos(k(R_2 - R)) + \\
    & 2 k R_2 \sin(k(R_2 - R)) - \\
    & (2 - k^2 R_1^2) \cos(k(R_1 - R)) - \\
    & 2 k R_1 \sin(k(R_1 - R))
    \Big].
    \end{split}
\end{equation}
