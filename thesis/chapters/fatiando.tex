\chapter{Fatiando a Terra}

\section{Resumen}


\section{Introducción}

% Necesidad de software open-source en geofísica.

Desde su invención, las computadoras han sido puestas a disposición de la
comunidad científica con el objetivo de resolver problemas que resultaban
inalcanzables.
Esta interacción entre una tecnología de vanguardia y el ambiente científico
generaba no solo beneficios para esta última parte, sino también una gran
retroalimentación.
Se desarrollaron lenguajes de programación especialmente diseñados para
resolver problemas numéricos junto con interfaces que facilitaran la
visualización y manipulación de datos científicos.
La relación entre ciencia y las herramientas computacionales se desarrolló tan
rápido que fue necesario crear el término \emph{computación científica} para
diferenciarla de los otros usos que se estaban gestando para las computadoras
(telecomunicaciones, fines comerciales, sistemas estatales de datos, etc.).
Hoy en día es imposible imaginar una ciencia que no necesite de las
herramientas computacionales para su avance y la resolución de los problemas
actuales que enfrenta.

A medida que los problemas científicos se vuelven cada vez más complejos de
resolver, las tareas científicas de aprender los últimos conocimientos en la
materia, adquirir nuevos datos, desarrollar el software necesario para
procesarlos y finalmente generar un nuevo conocimiento, se presenta como un
desafío titánico para ser desempeñado por una persona o por apenas un puñado de
investigadores.
La complejidad actual de la ciencia requiere que el clásico flujo de trabajo
científico se distribuya a lo ancho de la comunidad, ofreciendo productos
o soluciones para cada una de sus etapas, que puedan ser utilizados libremente
por otros investigadores y otras investigadoras, que a su vez puedan
modificarlos y volver a distribuirlos en caso de desearlo.
En resumen, los problemas científicos actuales requieren soluciones
comunitarias, tanto para dar respuestas a las preguntas fundamentales, así como
para desarrollar herramientas que faciliten la resolución de estos problemas.

Por fuera de la comunidad científica (aunque con algunas intersecciones) se
comenzó a gestar en la década del 80 un movimiento que trabajaba en la creación
de herramientas computaciones con características similares.

Movimiento open-source en ciencia (iPython, Jupyter, Numpy, Matplotlib,
Astropy, etc), estado actual.


\section{Historia}

Motivación inicial, personas involucradas, evolución del proyecto.

A partir de 2018 el proyecto tomó una nueva dirección.
El panorama de software de código abierto para Geofísica había cambiado mucho
desde los inicios de Fatiando a Terra: la cantidad de nuevos paquetes diseñados
para atacar diversos problemas de las geociencias había aumentado
considerablemente.
Dentro de este nuevo ecosistema, fatiando no presentaba un objetivo claro.
Esto no solo hacía difícil que potenciales usuarios y usuarias distingan el
propósito en sí del proyecto, sino que también constituía una base de código
difícil de mantener.
Por otro lado, para ese entonces fatiando había sido el hogar de
implementaciones de métodos muy novedosos (cuya audiencia está más orientada
a investigadores), así como de código \emph{juguete} diseñado para ser
utilizado en clases para fines didácticos, pero sin las capacidades para atacar
problemas reales.

Era necesario establecer nuevos objetivos del proyecto.
La decisión fue la de dividir el proyecto en múltiples paquetes con
objetivos claros y concisos.
Esto permitiría no sólo una fácil adopción por parte de usuarios y usuarias,
sino también que otros proyectos los utilicen como dependencias en caso de
desearlo.
Además, esta oportunidad abrió las puertas a hacer uso de mejores prácticas
para el desarrollo de software, estableciendo estándares de calidad más altos.
En la sección~\ref{sec:best-practices} se describen las mencionadas mejores
prácticas.

 dividir el fatiando en librerías más pequeñas.



\section{Paquetes de software}

Descripción de cada uno de los paquetes. Breve, casi como paper de JOSS.
Puedo poner códigos de ejemplo, especificar versiones.
Paper en JOSS de Verde y Pooch

\subsection{Verde}

\subsection{Boule}

\subsection{Harmonica}

\subsection{Pooch}


\section{Desarrollo y mejores prácticas}
\label{sec:best-practices}

Flujo de trabajo en Github.

Code review.

Documentación.

Testeo.

CI.

Ejemplos.

Filosofía UNIX?

Releases y backward compatibility?

Pypi y conda-forge?


\section{Comunidad}

Codigo de conducta

Instructivos para contribuir

Reuniones periodicas

Discusiones abiertas en issues/PRs

Interacción con otros proyectos?


\section{Adopción en la comunidad científica}

Publicaciones científicas relacionadas.

Tutoriales en Transform2020 y Transform 21 charla para la GSH

Visitas a la página web


\section{Planes a futuro}

Stable release of every package

Mayor interoperabilidad con el resto de paquetes geofísicos (SimPEG, pyGIMLi, etc)
