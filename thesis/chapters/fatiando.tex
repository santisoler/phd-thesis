\chapter{Fatiando a Terra}

\section{Resumen}


\section{Introducción}

Necesidad de software open-source en geofísica.

Movimiento open-source en ciencia (iPython, Jupyter, Numpy, Matplotlib,
Astropy, etc), estado actual.


\section{Historia}

Motivación inicial, personas involucradas, evolución del proyecto.


\section{Paquetes de software}

Descripción de cada uno de los paquetes. Breve, casi como paper de JOSS.
Puedo poner códigos de ejemplo, especificar versiones.
Paper en JOSS de Verde y Pooch

\subsection{Verde}

\subsection{Boule}

\subsection{Harmonica}

\subsection{Pooch}


\section{Desarrollo y mejores prácticas}

Flujo de trabajo en Github.

Code review.

Documentación.

Testeo.

CI.

Ejemplos.

Filosofía UNIX?

Releases y backward compatibility?

Pypi y conda-forge?


\section{Comunidad}

Codigo de conducta

Instructivos para contribuir

Reuniones periodicas

Discusiones abiertas en issues/PRs

Interacción con otros proyectos?


\section{Adopción en la comunidad científica}

Publicaciones científicas relacionadas.

Tutoriales en Transform2020 y Transform 21 charla para la GSH

Visitas a la página web


\section{Planes a futuro}

Stable release of every package

Mayor interoperabilidad con el resto de paquetes geofísicos (SimPEG, pyGIMLi, etc)
