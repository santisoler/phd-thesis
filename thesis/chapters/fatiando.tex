\chapter{Fatiando a Terra}

\section{Resumen}


\section{Introducción}

% Necesidad de software open-source en geofísica.

Desde su invención, las computadoras han sido puestas a disposición de la
comunidad científica con el objetivo de resolver problemas que resultaban
inalcanzables.
Esta interacción entre una tecnología de vanguardia y el ambiente científico
generaba no solo beneficios para esta última parte, sino también una gran
retroalimentación.
Se desarrollaron lenguajes de programación especialmente diseñados para
resolver problemas numéricos junto con interfaces que facilitaran la
visualización y manipulación de datos científicos.
La relación entre ciencia y las herramientas computacionales se desarrolló tan
rápido que fue necesario crear el término \emph{computación científica} para
diferenciarla de los otros usos que se estaban gestando para las computadoras
(telecomunicaciones, fines comerciales, sistemas estatales de datos, etc.).
Hoy en día es imposible imaginar una ciencia que no necesite de las
herramientas computacionales para su avance y la resolución de los problemas
actuales que enfrenta.

A medida que los problemas científicos se vuelven cada vez más complejos de
resolver, las tareas científicas de aprender los últimos conocimientos en la
materia, adquirir nuevos datos, desarrollar el software necesario para
procesarlos y finalmente generar un nuevo conocimiento, se presenta como un
desafío titánico para ser desempeñado por una persona o por apenas un puñado de
investigadores.
La complejidad actual de la ciencia requiere que el clásico flujo de trabajo
científico se distribuya a lo ancho de la comunidad, ofreciendo productos
o soluciones para cada una de sus etapas, que puedan ser utilizados libremente
por otros investigadores y otras investigadoras, que a su vez puedan
modificarlos y volver a distribuirlos en caso de desearlo.
En resumen, los problemas científicos actuales requieren soluciones
comunitarias, tanto para dar respuestas a las preguntas fundamentales, así como
para desarrollar herramientas que faciliten la resolución de estos problemas.

Por fuera de la comunidad científica (aunque con algunas intersecciones) se
comenzó a gestar en la década del 80 un movimiento que trabajaba en la creación
de herramientas computaciones con características similares.

Movimiento open-source en ciencia (iPython, Jupyter, Numpy, Matplotlib,
Astropy, etc), estado actual.


\section{Historia}

Los orígenes del proyecto se remonta a finales de la década del 2000, cuando
Leonardo Uieda, Vanderlei Oliveira Jr. junto a otros estudiantes se encontraban
cursando los últimos años del Bachillerato en Geofísica (\emph{Bacharelado em
Geofísica}) en la Universidade de São Paulo.
En este contexto surge la idea de implementar una alternativa propia a los
software de modelado gravimétrico 2D comerciales.
Tras asistir a cursos dictados por
Software Carpentry\footnote{%
    \url{https://software-carpentry.org}
}
en Canadá, Leonardo Uieda adquiere mayores conocimientos en el uso de sistema
de control de versiones junto a otras mejores prácticas para el desarrollo de
Software, regresa con nuevas ideas para el proyecto y un diagrama de un posible
diseño para este proyecto (fig.~\ref{fig:talwani-idea}).
La idea finalmente se desarrolla mediante
una implementación del método de \citet{talwani1959} para
modelado directo de polígonos 2D, y la posterior construcción de una interfaz
gráfica escrita en C que más tarde se reimplementaría en Python.

\begin{figure}
    \centering
    \includegraphics[width=\linewidth]{figs/fluxo-simples.pdf}
    \caption{
        Diagrama de flujo para la primera implementación de un software para el
        modelado gravimétrico de polígonos 2D. Realizado por Leonardo Uieda
        (circa 2009).
    }
    \label{fig:talwani-idea}
\end{figure}

En paralelo, y como proyecto final de su Bachillerato, Leonardo Uieda realiza
una implementación en Python del algoritmo para el cálculo de campos
gravitacionales de tesseroides mediante la \ac{GLQ}.
Luego de reescribirlo en C, este proyecto deriva en el lanzamiento del software
Tesseroids \citep{uieda2016} que ha sido ampliamente utilizado por la comunidad
geocientista.

El éxito de estos proyectos los llevó a aspirar a una idea mucho más
ambiciosa: desarrollar un software de código abierto para modelar el planeta
Tierra de
forma completa.
En ese entonces surge un nombre para el proyecto: \emph{Fatiando a Terra}, que
puede traducirse como \emph{rebanando la Tierra}.

Durante su Master en Geofisica en el Observatório Nacional, Rio de Janeiro,
Leonardo comienza el desarrollo de Fatiando a Terra, transformándolo en el hogar
de las implementaciones que realiza a lo largo de sus investigaciones, todas
mediante el lenguaje Python.
Entre ellas podemos encontrar: modelados directos con diferentes geometrías,
continuación ascendente, deconvolución de Euler, fuentes equivalentes, hasta un
incipiente \emph{framework} de inversión y algunas implementaciones de
tomografías sísmicas simples.

En los años posteriores, cuando la mayoría de los gestores del proyecto se
encontraban cursando sus Doctorados, Fatiando a Terra comienza a cobrar mayor
forma.
Leonardo y Vanderei deciden utilizar el \emph{framework} de inversión para dar
un curso sobre inversiones geofísicas en la Universidade de São Paulo.
Fatiando adquiere su propio dominio
(\href{https://www.fatiando.org}{fatiando.org}) y su primera página web,
alojada en un servidor casero por José Caparica Jr\footnote{%
    Más adelante se utilizarán los servicios de
    \href{https://readthedocs.org/}{Read The Docs} para alojar el sitio web,
    que más tarde se reemplazarían por GitHub Pages.
}.
Además se elige distribuirlo bajo la licencia
\href{https://opensource.org/licenses/BSD-3-Clause}{BSD 3-clause}.
En 2013, el proyecto es presentado en una charla en SciPy 2013
\citep{uieda2013}.

En los años posteriores el proyecto comienza a cobrar mayor reconocimiento.
Empieza a ser utilizado en publicaciones científicas
\citep[][entre otros]{%
    uieda2012,
    carlos2014,
    oliveira2015,
    hidalgogato2015,
    carlos2016,
    reis2016,
    uieda2017,
    hidalgogato2017,
    siqueira2017%
},
% fill in more papers if you know about some others
dictado de clases
(Tópicos de inversão em
geofísica\footnote{%
    \url{https://www.leouieda.com/teaching/inversao-iag-2012.html}
    y \url{https://www.leouieda.com/teaching/inversao-unb-2014.html}
},
\citet{uieda2014},
Geofísica 1: Gravimetria e magnetometria\footnote{%
    \url{https://www.leouieda.com/teaching/geofisica1.html}
},
Geofísica 2: Sismologia e sísmica\footnote{%
    \url{https://www.leouieda.com/teaching/geofisica2.html}
})
y en trabajos finales de grado y posgrado
\citep{carlos2013, sales2014, soler2015, uieda2016b, melo2020}.
Además comienza a atraer la atención de la comunidad internacional, recibiendo
colaboraciones de investigadores y desarrolladores de diferentes partes del
mundo.
La utilización de fatiando por otros investigadores e investigadoras deriva
a su vez en un ciclo de retroalimentación: quienes comienzan siendo usuarios
realizan colaboraciones.
De esta forma fatiando comienza a alojar implementaciones de métodos
novedosos recientemente publicados \citep{uieda2012b, oliveira2013}

Mi primea contribución al proyecto consistió en una implementación del promedio
radial del espectro de frecuencias de grillas de gravedad
o magnetismo\footnote{%
    \url{https://github.com/fatiando/fatiando/pull/303}
}.
Desde entonces comencé a involucrarme cada vez más en el proyecto, lo que me
permitió adquirir mayores conocimientos en el uso de controladores de
versiones, flujos de trabajo para el desarrollo colaborativo, creación de
funciones de \emph{testing}, buenas prácticas para el diseño de algoritmos y la
importancia de mantener la documentación actualizada.

La última versión del paquete fatiando es la v0.5, lanzada en Noviembre de
2016.
Si bien ese paquete en particular se encuentra obsoleto y no recibe mayor
mantenimiento, esto no significa que la vida de el proyecto haya finalizado en
ese entonces.

A partir de 2018 el proyecto tomó una nueva dirección.
El panorama de software de código abierto para Geofísica había cambiado mucho
desde los inicios de Fatiando a Terra: la cantidad de nuevos paquetes diseñados
para atacar diversos problemas de las geociencias había aumentado
considerablemente
\citep{cockett2015, ruecker2017, varga2019, obspy2019}.
Dentro de este nuevo ecosistema, fatiando no poseía un objetivo claro.
Esto no solo hacía difícil que potenciales usuarios y usuarias identifiquen el
propósito del proyecto, sino que también constituía una base de código difícil
de mantener.
Por otro lado, para ese entonces fatiando había sido el hogar de
implementaciones de métodos clásicos de la geofísica, métodos muy novedosos
(orientados principalmente a la investigación científica), así como de código
\emph{juguete} diseñado para ser utilizado en clases para fines didácticos
pero sin las capacidades para atacar problemas reales.
Sumado a esto, la versión de Python 2.7 llegaba pronto a su final de vida, lo
que hacía necesario adaptar fatiando al nuevo Python 3.

Estas razones ponían en evidencia la necesidad de establecer objetivos claros
para el proyecto, así como también repensar su diseño y sustentabilidad
a futuro.
Por esto se tomó la decisión de dividir el proyecto en varios paquetes que
posean objetivos claros y concisos.
Esto permitiría no sólo una fácil adopción por parte de usuarios y usuarias,
sino también que otros proyectos los utilicen como dependencias en caso de
desearlo.
Además, manteniendo los campos de acción de cada paquete aislados del resto,
se facilitaría el desarrollo a futuro: los colaboradores no necesitan
familiarizarse con el proyecto completo, sino solo con algunas de sus partes.
Por otro lado, la decisión de reescribir gran parte del código se presentó como
una oportunidad para pensar mejores diseños del software ya existente y de
implementar mejores
prácticas para el desarrollo de software, estableciendo estándares de calidad
más altos.


\section{Paquetes de software}

Descripción de cada uno de los paquetes. Breve, casi como paper de JOSS.
Puedo poner códigos de ejemplo, especificar versiones.
Paper en JOSS de Verde y Pooch

\subsection{Verde}

\subsection{Boule}

\subsection{Harmonica}

\subsection{Pooch}


\section{Desarrollo y mejores prácticas}
\label{sec:best-practices}

Flujo de trabajo en Github.

Code review.

Documentación.

Testeo.

CI.

Ejemplos.

Filosofía UNIX?

Releases y backward compatibility?

Pypi y conda-forge?


\section{Comunidad}

Codigo de conducta

Instructivos para contribuir

Reuniones periodicas

Discusiones abiertas en issues/PRs

Interacción con otros proyectos?


\section{Adopción en la comunidad científica}

Publicaciones científicas relacionadas.

Tutoriales en Transform2020 y Transform 21 charla para la GSH

Visitas a la página web


\section{Planes a futuro}

Stable release of every package

Mayor interoperabilidad con el resto de paquetes geofísicos (SimPEG, pyGIMLi, etc)
